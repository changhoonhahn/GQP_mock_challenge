\section{Introduction} \label{sec:intro} 
\todo{make the case for galaxy evolution with big data and where that has
brought the field so far} 

The Dark Energy Spectroscopic Instrument will conduct the largest spectroscopic
galaxy survey to date covering ${\sim}14,000~{\rm deg}^2$~\citep{desicollaboration2016}. 
Over the next five years, 

\todo{What is DESI? And why should galaxy folks care} 
Provide an overview of DESI specifics, numbers, and science goals, which will 
mostly be cosmology (BAO, RSD, etc). DESI will be great
Beyond its impact on cosmology, DESI will also be transformative for galaxy
science. 

\todo{Why do we need a VAGC} 
provide value-added galaxy
catalogs (VAGCs), which will be transformative for galaxy science. 
In the past, VAGCs such as the
MPA-JHU\footnote{https://wwwmpa.mpa-garching.mpg.de/SDSS/DR7/}
provided galaxy properties from emission line analyses of SDSS
spectra~\cite{brinchmann2004}. 

Similarly, the NYU-VAGC~\citep{blanton2005}, 

These catalogs have been indispensible for establishing the global statistical
view of galaxy properties~\citep[see][for a review]{blanton2009}. 
\todo{previous successes with value-added catalogs}
%Increasingly sophisticated statistical studies of the overall population of galaxies as a function of mass, cosmic time, and environment have provided a basic picture of the formation and evolution of galaxies

\todo{what are some key challenges that PROVABGS will address}

\todo{how will PROVABGS address those challenges?} 
extending previous statistcal analyses on SDSS to a larger sample and to higher
redshift. 
Mention some obvious ones: stellar mass function, luminosity function,
star-formation sequence.  
more detailed galaxy-halo connection 
mention some cosmological applications: multi-tracer analyses.  
Furthermore, since BGS will measure galaxy spectra down to $r\sim20$, we will
have faint dwarf galaxies at low redshift. 
provabgs introduces a new frontier of low signal-to-noise statistically
powerful sample that will require  

\todo{Why do we need a mock challenge? i.e. what are the goals of this paper?}
We want to test and cement our methodology specifically for our GQP 
analysis before SV data comes out. 
As part of the survey preparation, we have all the tools to accurately 
forward model observations. 
\todo{details on some of the specific tools and what we're able to simulate: 
realistic spectroscopy. realistic photometry. realistic spectro-photometry} 
All of this gives us a rare opportunity to test our methodology on bespoke
simulations. 

\todo{overview of the paper} 
A mock challenge is also great for testing new methodology.
BGS is a bright time survey and will push the boundaries of low SNR 
spectra. But if we can find a way to  infer robust galaxy properties the 
statistical payout is awesome. \todo{Something also about LRGs} 
We're also trying to robustly fit spectra and photometry simultaneously. 
This has been done before (\todo{citations}) but not extensively tested 
on simulations. 


\begin{figure}
\begin{center}
\includegraphics[width=\textwidth]{figs/bgs.pdf} 
\caption{DESI will conduct the largest spectroscopic survey to 
date covering ${\sim}14,000~{\rm deg}^2$. During dark time, DESI will measure
${>}20$ million spectra of luminous red galaxies, emission line galaxies, and 
quasars out to $z > 3$. During bright time, DESI will measure the spectra of 
${\sim}10$ million galaxies out to $z{\sim}0.4$  with the Bright Galaxy Survey (BGS).
{\em Left}: With its ${\sim}14,000~{\rm deg}^2$ footprint (black), DESI will 
cover ${\sim}2\times$ the SDSS footprint (blue; ${\sim}7000~{\rm deg}^2$) 
and ${\sim}45\times$ the GAMA footprint (orange; ${\sim}300~{\rm deg}^2$). 
{\em Right}: Over this footprint, the BGS will provide spectra for a magnitude 
limited sample of ${\sim}10$ million galaxies down to $r < 20$, two orders of 
magnitude deeper than the SDSS main galaxy sample and $0.2$ mag deeper than GAMA.}
\label{fig:bgs}
\end{center}
\end{figure}

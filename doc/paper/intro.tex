\section{Introduction} \label{sec:intro} 
% galaxy evolution with big data 
Large galaxy surveys have been transformational for our understanding of galaxy
evolution. 
With surveys such as the Sloan Digital Sky Survey~\citep[SDSS;][]{york2000},
Galaxy and Mass Assembly survey~\citep[GAMA;][]{driver2011}, and 
PRIsm MUlti-object Survey~\citep[PRIMUS;][]{coil2011}, 
we have now established the global trends of galaxies in the local universe. 
For instance, population statistiscs, such as the stellar mass
function~\citep{li2009, marchesini2009, moustakas2013} or quiescent
fraction~\citep{kauffmann2003a, blanton2003, baldry2006, taylor2009}, and their
evolution are now well understood. 
Many global scaling relations of galaxy propreties such as the mass-metallicity
relation~\citep{tremonti2004} or
the ``star formation sequence'' of star forming galaxies~\citep{noeske2007,
daddi2007, salim2007, speagle2014} have also been firmly established by these
observations. 
However, despite their importance in building our current understanding, these
empirical relations on their own do not directly shed light on how galaxies
form and evolve. 
%we've exhausted what we can do with the empirical relations of our current
%observations

Measuring the empirical relations with more precision and accuracy can
potentially reveal new nuanced trends among galaxies undetectable by previous
observations.
There are also new approaches that go beyond the observed relations.
Empirical prescriptions for physical proceses can be placed on top of $N$-body
simulations that capture hierarchical structure formation in empirical 
models~\citep[\emph{e.g.} {\sc UniverseMachine}][]{behroozi2019}. 
The predictions of these models can be compared to the observed distributions
of galaxy properties to derive insights into physical processes, such as the
timescale of star formation quenching~\citep{wetzel2013, hahn2017, tinker2017}. 
Predicted distributions of galaxy properties of large-scale cosmological 
hydrodyanmical simulations can also be compared to 
observations~\citep[\emph{e.g.}][]{genel2014, somerville2015a, dave2017a,
trayford2017, dickey2021, donnari2021}.
Advances in machine learning techniques for accelerating and emulating
hydrodynamical simulations~\citep{villaescusa-navarro2021} will enable such
comparisons to explore a broad range of galaxy formation models.
Soon we will be able to compare detailed galaxy formation models directly
against observations and explore the parameter spaces of the models. 
What all of these approaches share is that they require more statistically
poweful galaxy samples with well controlled systematics and well understood
selection function. 

The Dark Energy Spectroscopic Instrument (DESI) marks the next stage in large
galaxy surveys. 
Over the next five years, DESI will use its 5000 robotically-actuated fibers to
provide redshifts of ${\sim}30$ million galaxies over 
${\sim}14,000~{\rm deg}^2$, a third of the sky~\citep{desicollaboration2016,
desicollaboration2016a}.
The redshifts will be spectroscopically measured from optical spectra that
spans the wavelength range $3600 < \lambda < 9800\AA$ with spectral resolutions
$R = \lambda/\Delta \lambda = 2000 - 5000$.
In addition, DESI targets will also have photometry from the Legacy Imaging
Surveys Data Release 9~\citep[LS][]{dey2019}, used for target selection. 
LS is a combination of three public projects (Dark Energy Camera Legacy Survey,
Beijing-Arizona Sky Survey, and Mayall $z$-band Legacy Survey) that jointly
imaged the DESI footprint in three optical bands ($g$, $r$, and $z$). 
It also includes photometry in the \emph{Wide-field Infrared Survey Explorer}
$W1$, $W2$, $W3$, and $W4$ infrared bands.
The infrared photometry is derived from all imaging through year 4 of
NEOWISE-Reactivation force-photometered in the unWISE maps at the locations of
LS optical sources~\citep{meisner2017, meisner2017a}.

\begin{figure}
\begin{center}
\includegraphics[width=\textwidth]{figs/bgs.pdf} 
\caption{
    DESI will conduct the largest spectroscopic survey to date covering
    ${\sim}14,000~{\rm deg}^2$. 
    During dark time, DESI will measure ${>}20$ million spectra of luminous red
    galaxies, emission line galaxies, and quasars out to $z > 3$.
    During bright time, DESI will measure the spectra of ${\sim}10$ million
    galaxies out to $z{\sim}0.6$ with the Bright Galaxy Survey (BGS).
    {\em Left}: BGS (blue) will cover ${\sim}2\times$ the SDSS footprint
    (orange) and ${\sim}45\times$ the GAMA footprint (red).
    {\em Right}: We present the redshift distribution of BGS as predicted by
    the MXXL simulation (blue). 
    We include the redshift distribution of SDSS (orange) and GAMA multiplied
    by $10\times$ (red) for comparison. 
    BGS will be two orders of magnitude deeper than the SDSS main galaxy sample
    and deeper than GAMA.
    BGS will provide spectra for a magnitude limited sample of ${\sim}10$
    million galaxies down to $r < 19.5$ (BGS Bright) and a deeper sample of
    ${\sim 5}$ million galaxies as faint as $r < 20.175$ (BGS Faint).
}
\label{fig:bgs}
\end{center}
\end{figure}

During bright time, when the night sky is roughly ${\sim}2.5\times$ brighter
than the nominal dark codntions, DESI will conduct the Bright Galaxy Survey
(BGS). 
BGS will provide a $r < 19.5$ magnitude-limited sample of ${\sim}10$ million
galaxies out to redshift $z < 0.6$ --- the BGS bright sample. 
It will also provide a surface brightness and color selected sample of 
${\sim}5$ million faint galaxies with $19.5 < r < 20.175$. 
The completeness and effect of systematics of the BGS galaxy samples are
characterized in detail in Hahn~\etal~(in prep.) using observations from
DESI Survey Validation. 
%From the BGS galaxy samples we will be able to measure the Baryon Acoustic Oscillation to the cosmic variance limit. 
%They also have the potential to unlock the constraining power of galaxy clustering with higher-order statistics and multiple-tracer methods~\citep{desicollaboration2016, desicollaboration2016a}.
%Beyond cosmology, BGS will be transformative for galaxy evolution. 
Compared to the seminal SDSS main galaxy survey, BGS will provide optical
spectra two magnitudes deeper, over twice the sky, and ${\sim}4$ billion years
farther (Figure~\ref{fig:bgs}). 
It will observe a broader range of galaxies than previous surveys with
unprecendented statistical power. 
Hence, BGS offers a unique opportunity to apply more sophisticated statistical
analyses and new approaches to reveal the detailed connections among galaxy
populations and  advance galaxy evolution studies. 

\begin{figure}
\begin{center}
    \includegraphics[width=0.5\textwidth]{figs/bgs_mstar_z.pdf}
    \caption{
        Stellar mass ($M_*$) distribution as a function of redshift of the $r < 19.5$
        magnitude-limited BGS Bright sample (orange) as predicted by the MXXL
        simulation. 
        We include the $M_*$ distribution of MXXL galaxies with $r < 20$ (blue)
        for reference.
        Many such fainter galaxies will be included in the BGS Faint sample,
        which will observe galaxies as faint as $r < 20.175$. 
        BGS will observe a wide range of galaxies with high completeness and
        unprecedented statistical power.
        \emph{We will apply state-of-the-art Bayesian SED modeling to all BGS
        galaxies to construct the PRObabilistic Value-Added BGS (PROVABGS)
        catalog, which will unlock new and more sophisticated approaches to
        galaxy evolution studies.}}\label{fig:bgs_mstar}
\end{center}
\end{figure}

Observations alone, however, are not sufficient.  
New techniques and approaches require robust and consistent measuremnts of
galaxy properties from the observations.
The many advantages of the BGS observations would be squandered if they were 
analyzed inconsistently with a hodgepodge of methodologies.  
Value-added catalogs (VACs) that provide consistently measured galaxy
properties for entire galaxy surveys are instrumental in this regard and have
been used by hundreds of scientific
analyses~\citep[see][for a review]{blanton2009}. 
For SDSS galaxies, the NYU-VAGC~\citep{blanton2005} provides photometric properties
(\emph{e.g.} absolute magnitudes) and the MPA-JHU
catalog~~\citep{brinchmann2004}\footnote{https://wwwmpa.mpa-garching.mpg.de/SDSS/DR7/}
provides spectral properties (\emph{e.g.} emission line
luminosities).
Despite being released over a decade ago, these VACs are still widely used
today~\citep[\emph{e.g.}][]{alpaslan2021, odonnell2021, trevisan2021}. 
Accompanying the BGS observations, the DESI Galaxy Quasar Physics (GQP) working
group will produce the PRObabilistic Valued-Added Bright Galaxy Survey
(PROVABGS) catalog. 

For all ${>}10$ million BGS galaxies, PROVABGS will provide full posterior
probability distributions of physical properties such as stellar mass ($M_*$),
star formation rate (SFR), metallicity ($Z$), and stellar age ($\tage$). 
These properties will be inferred from both the LS photometry and DESI
spectroscopy using a state-of-the-art Bayesian modeling of the galaxy spectral
energy distribution (SED). 
PROVABGS will enable conventional analyses to be extended to a more
statistically powerful spectroscopic galaxy sample. 
Population statistics such as the stellar mass function or the star formation
sequence will be measured with higher precision than previously possible and
for a wider range of galaxies (Figure~\ref{fig:bgs_mstar}). 
The high completeness and simple selection function of the BGS Bright sample
will also facilitate  comparisons to empirical models or galaxy formation
simulations with new approaches. 

Moreover, PROVABGS will provide more accurate measurements of galaxy properties
from full posterior distributions, rather than point estimates. 
Posteriors more accurately estimate the uncertainties on the galaxy property
measurements and any degeneracies among them. 
Hence, PROVABGS will enable a new level of statistical robustness in galaxy
evolution studies.
The PROVABGS posteriors will also open the door for Bayesian Hierarchical
approaches.
For instance, we can combine posteriors of galaxies to conduct principled
population inference. 
Beyond mitigating biases, this will allow us to more fully exploit the 
observations since we can robustly include galaxies with less tightly
constrained properties.  
Little explored low signal-to-noise regimes that can be probed with this
approach may shed light on galaxy evolution.
Hierarchical inference can also improve the statistical power of BGS through
Bayesian shrinkage: the joint posterior of the galaxy sample can be used as the
prior to shrink the uncertainties on the properties of individual galaxies. 
With all these advantages, PROVABGS will provide a VAC that fully exploits the
DESI observations and maximizes the scientific impact of BGS. 

In this paper, we present the mock challenge for PROVABGS conducted by the GQP
working group. 
We present the state-of-the-art SED modeling that will be used to infer the
galaxy properties of BGS galaxies and construct the PROVABGS. 
We use an SED model with non-parametric prescriptions for galaxy star formation
and metallicity histories and accelerate the parameter inference using neural
emulators. 
Moreover, we validate our SED modeling on realistic mock BGS observations
constructed using the {\sc L-Galaxies} semi-analytic
model~\citep{henriques2015} and DESI survey simulations. 
By applying our SED model on mock observations, where we know the true galaxy
properties, we demonstrate that we can accurately infer galaxy properties for
PROVABGS and highlight the advantages of jointly analyzing photometric and
spectroscopic observations. 
Furthermore, we characterize, in detail, the limitations of our SED modeling so
that future studies using PROVABGS can use this work as a key reference in
interpreting their results. 

In Section~\ref{sec:sims}, we describe the {\sc L-Galaxies} semi-analytic
model and how we use them to construct mock BGS observations. 
We then present the SED model, our Bayesian parameter inference framework with
neural emulators, and the mock challenge in Section~\ref{sec:methods}. 
We present the results of the mock challenge in Section~\ref{sec:results} and
discuss their implications in Section~\ref{sec:discuss}. 

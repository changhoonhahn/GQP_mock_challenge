\section{Introduction} \label{sec:intro} 
% galaxy evolution with big data 
Large galaxy surveys have been transformational for our understanding of galaxy
evolution. 
With surveys such as the Sloan Digital Sky Survey~\citep[SDSS][]{york2000},
Galaxy and Mass Assembly survey~\citep[GAMA][]{driver2011}, and 
PRIsm MUlti-object Survey~\citep[PRIMUS][]{coil2011}, 
we have now established the global trends of galaxies in the local universe. 

%Observations from large surveys such as the Sloan Digital Sky Survey (SDSS; York et al. 2000) have been critical for establishing the global trends of galaxies in the local universe. Broadly speaking, galaxies fall into two categories: quiescent and star-forming (hereafter SF) galaxies. Quiescent galaxies have little to no star formation, are red in color due to old stellar populations, and have elliptical morphologies. Meanwhile, SF galaxies have significant star formation, thus are blue in color, and have disk-like morphologies (Kauffmann et al. 2003; Blanton et al. 2003; Baldry et al. 2006; Taylor et al. 2009; Moustakas et al. 2013; see Blanton & Moustakas 2009 and references therein). SF galaxies, furthermore, are found on the so-called “star-forming sequence” (hereafter SFS), a tight relationship between their star formation rates (SFR) and stellar masses (Noeske et al. 2007; Daddi et al. 2007; Salim et al. 2007; Speagle et al. 2014; Lee et al. 2015, see also Figure 1). This sequence, which is observed out to z > 2 (Wang et al. 2013; Leja et al. 2015) plays a crucial role in determining galaxy evolution over the past ∼10Gyr (see Kelson 2014; Abramson et al. 2016, for an alternative point of view). The significant fraction of SF galaxies that quench their star formation and migrate off of the SFS reflects the growth in the fraction of quiescent galaxies (Blanton 2006; Borch et al. 2006; Bundy et al. 2006; Moustakas et al. 2013). The decline of star formation in the entire SFS (Lee et al. 2015; Schreiber et al. 2015) over time reflects the decline in overall cosmic star formation (Hopkins & Beacom 2006; Behroozi et al. 2013; Madau & Dickinson 2014). With its evolution, the SFS also connects the star formation histories of SF galaxies to their stellar mass growths.



The Dark Energy Spectroscopic Instrument (DESI) marks the next stage in large
galaxy surveys. 
Over the next five years, DESI will use its 5000 robotically-actuated fibers to
provide redshifts of ${\sim}30$ million galaxies over 
${\sim}14,000~{\rm deg}^2$, a third of the sky~\citep{desicollaboration2016}.
The redshifts will be spectroscopically measured from optical spectra that
spans the wavelength range $3600 < \lambda < 9800\AA$ with spectral resolutions
$R = \lambda/\Delta \lambda = 2000 - 5000$.
In addition, DESI targets will also have photometry from the Legacy Imaging
Surveys Data Release 9~\citep[LS][]{dey2019}, used for target selection. 
LS is a combination of three public projects (Dark Energy Camera Legacy Survey,
Beijing-Arizona Sky Survey, and Mayall $z$-band Legacy Survey) that jointly
imaged the DESI footprint in three optical bands ($g$, $r$, and $z$). 
It also includes photometry in the WISE $W1$, $W2$, $W3$, and $W4$ infrared bands.
\todo{The infrared photometry is from all imaging through year 4 of
NEOWISE-Reactivation force-photometered in the unWISE maps at the locations of
LS optical sources (cite).}




\begin{figure}
\begin{center}
\includegraphics[width=\textwidth]{figs/bgs.pdf} 
\caption{
    DESI will conduct the largest spectroscopic survey to date covering
    ${\sim}14,000~{\rm deg}^2$. 
    During dark time, DESI will measure ${>}20$ million spectra of luminous red
    galaxies, emission line galaxies, and quasars out to $z > 3$.
    During bright time, DESI will measure the spectra of ${\sim}10$ million
    galaxies out to $z{\sim}0.6$ with the Bright Galaxy Survey (BGS).
    {\em Left}: DESI BGS (blue) will cover ${\sim}2\times$ the SDSS footprint
    (orange; ${\sim}7000~{\rm deg}^2$) and ${\sim}45\times$ the GAMA footprint
    (red; ${\sim}300~{\rm deg}^2$).
    {\em Right}: We present the redshift distribution of DESI BGS as predicted
    by the MXXL simulation (blue).  
    We include the redshift distribution of SDSS (orange) and GAMA multiplied
    by $10\times$ (red) for comparison. 
    BGS will be two orders of magnitude deeper than the SDSS main galaxy sample
    and deeper than GAMA.
    BGS will provide spectra for a magnitude limited sample of ${\sim}10$
    million galaxies down to $r < 19.5$ (BGS Bright) and a deeper sample of
    galaxies as faint as $r < 20.175$ (BGS Faint).
}
\label{fig:bgs}
\end{center}
\end{figure}

During bright time (\emph{e.g.} when the moon is above the horizon), DESI will
conduct the Bright Galaxy Survey (BGS). 
BGS will provide a $r < 19.5$ magnitude-limited sample of ${\sim}10$ million
galaxies out to redshift $z < 0.6$ --- \emph{i.e.} the BGS bright sample. 
It will also provide a surface brightness and color selected sample of 
\todo{X faint galaxies with $19.5 < r < 20.1$}. 
With these BGS galaxy samples we will be able to measure the Baryon Acoustic
Oscillation to the cosmic variance limit. 
They also have the potential to unlock higher-order statistics and
multiple-tracer methods. 
\todo{overview of cosmology impact of BGS}~\citep{desicollaboration2016,
desicollaboration2016a}.


\begin{figure}
\begin{center}
    \includegraphics[width=0.5\textwidth]{figs/bgs_mstar_z.pdf}
    \caption{
        Stellar mass ($M_*$) distribution as a function of redshift of the $r < 19.5$
        magnitude-limited BGS Bright sample (orange) as predicted by the MXXL
        simulation. 
        We include the $M_*$ distribution of MXXL galaxies with $r < 20$ (blue)
        for reference.
        Many such fainter galaxies will be included in the BGS Faint sample,
        which will observe galaxies as faint as $r < 20.175$. }\label{fig:bgs_mstar}
\end{center}
\end{figure}

BGS will be equally transformative for galaxy evolution. 
Compared to the seminal SDSS main galaxy survey, BGS will provide optical
spectra two magnitudes deeper, over twice the sky, and \todo{X} billion years
farther. 
This provides a unique opportunity to 
\todo{develop a more complete picture of low mass galaxies. 
Extend detailed galaxy environment studies to higher redshifts. 
Overall more complete picture of galaxy evolution 
}

Alongside the BGS, the DESI collaboration will also provide a number of
Value-Added Catalogs (VAC) with detailed properties of BGS galaxies to maximize
the scientific impact of DESI. 
In the past, such VACs have been instrumental for hundreds of scientific
analyses~\citep[see][for a review]{blanton2009}. 
The New York University-Value-Added Galaxy
Catalog~\citep[NYU-VAGC][]{blanton2005}, for instance, provided photometric
properties such as absolute magnitudes of SDSS galaxies. 
The MPA-JHU catalog\footnote{https://wwwmpa.mpa-garching.mpg.de/SDSS/DR7/}
provided spectral properties such as emission line
luminosities~\cite{brinchmann2004}.
Despite being released over a decade ago, these VACs are still widely used
today~\citep[\emph{e.g.}][]{alpaslan2021}. 

One key VAC that DESI will provide is the PRObabilistic Valued-Added Bright
Galaxy Survey (PROVABGS) catalog. 
For all ${\sim}10$ million BGS galaxies, PROVABGS will provide posterior
probability distributions of physical properties such as stellar mass ($M_*$),
star formation rate (SFR), metallicity, and stellar age. 
These properties will be inferred from the LS photometry and DESI spectroscopy
using a state-of-the-art Bayesian modeling of the galaxy spectral energy
distribution (SED). 
With these properties, PROVABGS will enable direct extension of previous works
to the more complete and statistically powerful spectroscopic galaxy sample. 
For instance, $M_*$ values will be used to measure the stellar mass function
down to \todo{$M_* \sim 10^{8}M_\odot$} (Hahn\etal~in prep.). 
$M_*$ and SFR will also be used to more precisely measure the star forming
sequence.
Detailed galaxy environment studies will be 

Moreover, instead of point estimates of galaxy properties, PROVABGS will
provide full posterior distributions. 
\todo{provabgs introduces a new frontier of low signal-to-noise statistically
powerful sample that will require}

%Increasingly sophisticated statistical studies of the overall population of galaxies as a function of mass, cosmic time, and environment have provided a basic picture of the formation and evolution of galaxies

\todo{Why do we need a mock challenge? i.e. what are the goals of this paper?}
We want to test and cement our methodology specifically for our GQP 
analysis before SV data comes out. 
As part of the survey preparation, we have all the tools to accurately 
forward model observations. 
\todo{details on some of the specific tools and what we're able to simulate: 
realistic spectroscopy. realistic photometry. realistic spectro-photometry} 
All of this gives us a rare opportunity to test our methodology on bespoke
simulations. 

\todo{overview of the paper} 
A mock challenge is also great for testing new methodology.
BGS is a bright time survey and will push the boundaries of low SNR 
spectra. But if we can find a way to  infer robust galaxy properties the 
statistical payout is awesome. \todo{Something also about LRGs} 
We're also trying to robustly fit spectra and photometry simultaneously. 
This has been done before (\todo{citations}) but not extensively tested 
on simulations. 



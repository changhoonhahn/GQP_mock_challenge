\begin{figure}
\begin{center}
\includegraphics[width=0.9\textwidth]{figs/inferred_props.pdf}
\caption{
    Comparison between the true galaxy properties, $\theta_{\rm true}$, and
    those inferred from SED modeling of mock observations, $\hat{\theta}$. 
    From the left to right columns, we compare $\log M_*$, $\log \avgsfr$, 
    $\log Z_{\rm MW}$, $\tage$ and $\tauism$. 
    The inferred galaxy properties are derived from SED modeling of mock
    spectra (top), photometry (middle), and spectrophotometry (bottom). 
    For each simulated galaxy, we represent the marginalized posterior of
    $\theta$ with a violin plot.  %We derive unbiased and precise $\log M_*$
    \emph{The posteriors demonstrate that, overall, we can derive accurate and 
    precise constraints on certain galaxy properties from joint SED modeling of
    DESI photometry and spectra.}
    } \label{fig:prop_inf}
\end{center}
\end{figure}

\section{Results} \label{sec:results}
%\subsection{Inferred Galaxy Properties}
The goal of this work is to demonstrate the precision and accuracy of inferred
galaxy properties for PROVABGS. 
We apply our SED modeling to the mock observables of \todo{100} \lgal~galaxies.
From the posterior distributions of the SPS parameters, we derive the following
physical galaxy properties: stellar mass ($M_*$), SFR averaged over 1 Gyr
($\avgsfr$), mass-weighted stellar metallicity ($Z_{\rm MW}$), and diffuse-dust
optical depth ($\tau_{\rm ISM}$).
$M_*$ and $\tau_{\rm ISM}$ are both SPS model parameters. 
We derive $\avgsfr$ and $Z_{\rm MW}$ as 
\begin{equation}
    \avgsfr = \frac{\int\limits_{t_{\rm age} - {\rm 1 Gyr}}^{t_{\rm age}}{\rm
    SFH}(t)\,{\rm d}t}{{\rm 1 Gyr}} \quad{\rm and}\quad
    Z_{\rm MW} = \frac{\int\limits_0^{t_{\rm age}}{\rm SFH}(t)\,{\rm
    ZH}(t)\,{\rm d}t}{M_*}.
\end{equation} 

In Figure~\ref{fig:prop_inf}, we compare the galaxy properties inferred from
SED modeling the mock observations, $\hat{\theta}$ to the true (input) galaxy
properties, $\theta_{\rm true}$ of the simulated galaxies.
In each column, we compare $\log M_*$, $\log \avgsfr$, $\log Z_{\rm MW}$,
$\tage$, and $\tauism$ from left to right.  
The inferred properties in the top, middle, and bottom rows are derived from
SED modeling of spectra, photometry, and spectrophotometry, respectively.
In each panel, we plot $\hat{\theta}$ using a violin plot, where the width
of the marker represents the marginalized posterior distribution of $\theta$. 
We note that in the comparison with SED modeling spectra only, we do not
include $f_{\rm fiber}$. 
Therefore, the true stellar mass in this case corresponds to $f_{\rm fiber}
\times M_*$. 
\emph{Overall, the comparison demonstrates that we can robustly infer galaxy
properties using the {\sc PROVABGS} SED modeling}. 

In more detail, we find that we infer unbiased and precise constraints on
$M_*$ throughout the entire $M_*$ range. 
%For spectra+photometry SED modeling, the posteriors have $\sigma_{M_*}\sim0.06$ dex.
We also infer robust $\avgsfr$ above $\log \avgsfr > -1$ dex; below this limit,
however, the inferred $\avgsfr$ are significantly less precise and
overestimate the true $\avgsfr$. 
This bias at low $\avgsfr$ is caused by model priors, which we discuss in
further detail later in \todo{Section~\ref{sec:discuss} and
Appendix~\ref{sec:model_priors}}. 
Both $Z_{\rm MW}$ and $\tage$ are not precisely constrained. 
However, Figure~\ref{fig:prop_inf} does not exhibit clear biases in the
constraints.
For $\tage$, the posteriors reveal the log-spaced $\tlb$ binning used in our
SPS model for $\tage > 6$ Gyr.
Lastly, $\tauism$ is overally accurately inferred for galaxies with low
$\tauism$ but appears to be underestimated for for high $\tauism$.

The overall constraints on galaxy properties for the mock observations is
especially encouraging due to the significant differences in the forward
model used to generate the observations and the SPS model used in the SED
modeling. 
First, the SFHs in the mock observations are taken directly from \lgal~simulation
outputs while the SFH parameterization in the SPS model is based on NMF bases fit
to IllustrisTNG galaxy SFHs.
Second, in the forward model, we construct the SED of the bulge and disk
components of the simulated galaxies separately: the components have separate
SFHs and ZHs. 
Lastly, we use different dust prescriptions: \cite{mathis1983} dust attenuation  
curve in the forward model and \cite{kriek2013} dust attenuation curve in the
SPS model. 
Despite these significant differences, our constraints on certain galaxy
properties are unbiased and precise. 

Figure~\ref{fig:prop_inf}, also highlights the advantages of jointly modeling
spectra and photometry. 
Comparing the constraints from spectrophotometry to photometry alone, we find
that including spectra significantly tightens the constraints for all
properties. 
In addition to tightening constraints, including spectra also appears to reduce
biases of the constraints. 
For instance, with only photometry, we derive significantly more biased
$\avgsfr$ constraints.
This is due to the limited constraining power of photometry, which allows the
posteriors to be dominated by model priors. 
Adding spectra, significantly increases the contribution of the likelihood and
ameliorates this effect. 
%Therefore, joint SED modeling of spectra and photometry 

Beyond qualitative comparisons of the posterior, we want to quantify the
precision and accuracy of the inferred galaxy properties. 
Let $\Delta_{\theta,i}$ be the discrepancy between the inferred and true
parameters for each galaxy: 
$\Delta_{\theta,i} = \hat{\theta}_i - \theta^{\rm true}_i$.
Then, if we assume that $\Delta_{\theta,i}$ are sampled from a Gaussian
distribution,
\begin{equation} \label{eq:eta_gauss}
    \Delta_{\theta,i} \sim \mathcal{N}(\mu_{\Delta_{\theta}}, \sigma_{\Delta_{\theta}}),
\end{equation}
the mean ($\mu_{\Delta_{\theta}}$) and standard deviation
($\sigma_{\Delta_{\theta}}$) of the distribution are population hyperparameters
that represent the accuracy and precision of the inferred posteriors for the
galaxy population. 
We can derive $\mu_{\Delta_{\theta}}$ and $\sigma_{\Delta_{\theta}}$ using a
hierarchical Bayesian framework~\citep[\emph{e.g.}][]{hogg2010,
foreman-mackey2014, baronchelli2020}.

Let $\{{\bfi X}_i\}$ represent the photometry or spcetrum of a galaxy
population and $\eta_\Delta = \{\mu_{\Delta_{\theta}},
\sigma_{\Delta_{\theta}}\}$ be the population hyperparameters.
Our goal is to constrain $\eta_\Delta$ from $\{{\bfi X}_i\}$ --- \emph{i.e.}
$p(\eta_\Delta \given \{{\bfi X}_i\})$.
As usual, we can expand 
\begin{align}
p(\eta_\Delta \given \{{\bfi X_i}\}) 
    =&~\frac{p(\eta_\Delta)~p( \{{\bfi X_i}\} \given \eta_\Delta)}{p(\{{\bfi X_i}\})}\\
    =&~\frac{p(\eta_\Delta)}{p(\{{\bfi X_i}\})}\int p(\{{\bfi X_i}\} \given \{\theta_i\})~p(\{\theta_i\} \given \eta_\Delta)~{\rm d}\{\theta_i\}.
\intertext{
    $\theta_i$ is the SPS parameters for galaxy $i$ and $p(\{{\bfi X_i}\}
    \given \{\theta_i\})$ is likelihood of the set of observations $\{{\bfi
    X_i}\}$ given the set of $\{\theta_i\}$. 
    Since the likelihoods for each of the $N$ galaxies, $p(\bfi X_i \given
    \theta_i)$, are not correlated, we can factorize and write the expression
    above as 
}
    =&~\frac{p(\eta_\Delta)}{p(\{{\bfi X_i}\})}\prod\limits_{i=1}^N\int p({\bfi X_i} \given \theta_i)~p(\theta_i \given \theta_\Delta)~{\rm d}\theta_i\\
    =&~\frac{p(\eta_\Delta)}{p(\{{\bfi X_i}\})}\prod\limits_{i=1}^N\int
    \frac{p(\theta_i \given {\bfi X_i})~p({\bfi X_i})}{p(\theta_i)}~p(\theta_i
    \given \eta_\Delta)~{\rm d}\theta_i\\
    =&~p(\eta_\Delta)\prod\limits_{i=1}^N\int \frac{p(\theta_i \given {\bfi
    X_i})~p(\theta_i \given \eta_\Delta)}{p(\theta_i)}~{\rm d}\theta_i. 
\intertext{
    $p(\theta_i \given {\bfi X_i})$ is the posterior for an individual galaxy,
    so the integral can be estimated using the Monte Carlo samples from the
    posterior: 
}
    \approx&~p(\eta_\Delta)\prod\limits_{i=1}^N\frac{1}{S_i}\sum\limits_{j=1}^{S_i}
    \frac{p(\theta_{i,j} \given \eta_\Delta)}{p(\theta_{i,j})}.
\end{align} 
$S_i$ is the number of posterior samples and $\theta_{i,j}$ is the $j^{\rm th}$
sample of galaxy $i$.
In practice,
$p(\theta_{i,j} \given \eta_\Delta) = p(\Delta_{\theta,i,j} \given
\eta_\Delta)$ is a Gaussian distribution and, hence, easy to evaluate. 
$p(\theta_{i,j}) = 1$ since we use uninformative and Dirichlet priors
(Table~\ref{tab:params}). 
Lastly, we sample the $p(\eta_\Delta \given \{{\bfi X_i}\})$ distribution using
MCMC.

\begin{figure}
\begin{center}
    \includegraphics[width=0.85\textwidth]{figs/etas_v2.pdf}
    \caption{
        The accuracy and precision of galaxy property posteriors from our
        joint SED modeling of spectrophotometry, quantified using population
        hyperparameters $\mu_{\Delta_{\theta}}$ and $\sigma_{\Delta_{\theta}}$,
        as a function of true galaxy property (green). 
        We plot $\theta_{\rm true} + \mu_{\Delta_{\theta}}$ in solid line and
        represent $\sigma_{\Delta_{\theta}}$ with the shaded region.
        We include $\{\mu_{\Delta_{\theta}}, \sigma_{\Delta_{\theta}}\}$ for SED
        modeling of photometry alone (orange) for comparison. 
        Including DESI spectra significantly improves both the accuracy and
        precision of the inferred galaxy properties. 
        $\log\avgsfr$, $\log Z_{\rm MW}$, and $\tage$ constraints are significantly
        impacted by well-characterized priors imposed by our SPS model.
        Meanwhile, discrepancies in the dust prescriptions between our SPS
        model and the mock observations drive the bias in $\tauism$.
        Nevertheless, \emph{
            we accurately and precisely infer: 
            $\log M_*$ for all $M_*$, $\log\avgsfr$ above $\log\avgsfr > -1\,{\rm
            dex}$, and $\tage$ below $8\,{\rm Gyr}$.
        }
        } \label{fig:etas}
\end{center}
\end{figure}

In Figure~\ref{fig:etas}, we present the accuracy ($\mu_{\Delta_{\theta}}$) and
precision ($\sigma_{\Delta_{\theta}}$) of our joint SED modeling of spectra and
photometry (green) as a function of true galax property. 
In each panel, we derive $p(\eta_\Delta \given \{{\bfi X_i}\})$ for 
$\log M_*$, $\avgsfr$, $\log Z_{\rm MW}$, $\tage$, and $\tauism$ in bins of
widths 0.2 dex, 0.5 dex, 0.05 dex, 0.5 Gyr, and 0.1, respectively. 
We only include bins with more than ten galaxies. 
$\mu_{\Delta_{\theta}}$ (solid) and $\sigma_{\Delta_{\theta}}$ (shaded region)
are the median values of $p(\mu_{\Delta_{\theta}}, \sigma_{\Delta_{\theta}}
\given \{{\bfi X_i}\})$ posterior. 
For comparison, we include $\{\mu_{\Delta_{\theta}},
\sigma_{\Delta_{\theta}}\}$ for SED modeling of photometry alone (orange).
We also include $\log\zmw$ $\{\mu_{\Delta_\theta}, \sigma_{\Delta_\theta}\}$
for simulated galaxies with $r_{\rm fiber} > 20$ (black dot-dashed) and
$\tauism$ $\{\mu_{\Delta_\theta}, \sigma_{\Delta_\theta}\}$ for galaxies
without bulges (black dotted), which we discuss later. 

\begin{figure}
\begin{center}
    \includegraphics[width=0.95\textwidth]{figs/etas_photo.pdf}
    \caption{
        Accuracy and precision of the galaxy properties inferred from joint SED
        modeling of spectra+photometry as a function of $r_{\rm fiber}$, $r$,
        $g-r$, and $r-z$.
        $r_{\rm fiber}$ and $r$ magnitudes are proxies for SNR of spectra and
        photometry. 
        $g-r$ and $r-z$ are the optical photometric colors. 
        In the top to bottom rows, we present $\eta_\Delta$ for $\log M_*$,
        $\log\avgsfr$, $\log Z_{\rm MW}$, $t_{\rm age, MW}$ and $\tau_{\rm ISM}$.
        We find a significant dependence on spectral SNR in the inferred 
        $\log \zmw$. 
        When the spectral SNR is low ($r_{\rm fiber} > 20$), the prior on 
        $\log \zmw$ imposed by our SPS model dominate the posterior and
        cause us to overestimate $Z_{\rm MW}$. 
        We find a significant color dependence on $\log\avgsfr$, $\log Z_{\rm
        MW}$, and $\tage$. 
        The color dependence on the $\log\zmw$ and $\tage$ constraints are
        driven by underlying correlations with spectral SNR and true $\tage$. 
        Meanwhile, $\log\avgsfr$ is overestimated for the reddest galaxies with
        $r - z > 0.6$, which correspond to quiescent galaxies with $\log\avgsfr
        < -1$ dex. 
        For the rest, we find no significant dependence on SNR or optical
        color. 
    }    
    \label{fig:eta_photo}
\end{center}
\end{figure}

In Figure~\ref{fig:eta_photo}, we examine how the accuracy and precision of
our galaxy parameter constraints are impacted by signal-to-noise ratio (SNR) or
photometric color. 
We present $\eta_\Delta$ of our joint SED modeling of spectra and photometry as
a function of $r_{\rm fiber}$, $r$, $g-r$, and $r-z$. 
$r_{\rm fiber}$ and $r$ magnitudes serve proxies of the SNR for the spectra and
photometry, respectively. 
In each row, we plot $\eta_\Delta$ for a different galaxy property: $\log M_*$,
$\avgsfr$, $\log Z_{\rm MW}$, $\tage$ and $\tauism$ (from top to bottom).

\begin{figure}
\begin{center}
    \includegraphics[width=\textwidth]{figs/etas_msfr.pdf} 
    \caption{
        Accuracy and precision of the galaxy properties inferred from joint SED
        modeling of spectrophotometry as a function of the galaxies' true $M_*$
        and $\avgsfr$. 
        We present $\mu_{\Delta_{\theta}}$ and $\sigma_{\Delta_{\theta}}$ in
        ($M_*$, $\avgsfr$) bins for $\log M_*$, $\log\avgsfr$, $\log Z_{\rm
        MW}$, $t_{\rm age, MW}$ and $\tau_{\rm ISM}$ in the top and bottom
        panels respectively. 
    }\label{fig:etas_msfr}
\end{center}
\end{figure}

Lastly, in Figure~\ref{fig:etas_msfr}, we investigate whether there are any
underlying systematic dependence in the inferred galaxy properties on the  
$M_*$-${\rm SFR}$ plane. 
In the top and bottom panels, we present $\mu_{\Delta_{\theta}}$ and 
$\sigma_{\Delta_{\theta}}$ in $(\log M_*, \log\avgsfr)$ bins for 
$\log M_*$, $\log\avgsfr$, $\log Z_{\rm MW}$, $\tage$ and $\tauism$ (left to
right).
We use $\log M_*$ bins of 0.225 dex width and $\log\avgsfr$ bins of width 
0.25 dex for $\log \avgsfr > 0$ dex and 0.5 dex for $\log \avgsfr < 0$ dex. 
We only present bins with more than 10 galaxies. 
On the $M_*-{\rm SFR}$ plane, we can examine whether the accuracy and precision
of the inferred properties have significant dependencies for galaxy type.\\ 

% log M*
\noindent \underline{\emph{Inferred $\log M_*$}}: 
Overall, we infer accurate and precise $\log M_*$ from our {\sc PROVABGS} SED
modeling. 
First, there is no significant dependence in $\mu_{\Delta_{\theta}}$ and 
$\sigma_{\Delta_{\theta}}$ with $\log M_*$ throughout the $\log M_*$ range. 
This means that we accurately infer the true $M_*$ throughout ${\sim}10^{9} -
10^{12} M_\odot$  with uniform precision of $\sigma_{\Delta_{\log M_*}} \sim 0.1$ dex. 
We also find no significant dependence on SNR dependence for $M_*$ ---
neither $r_{\rm fiber}$ nor $r$ magnitudes affect $\mu_{\Delta_{\log M_*}}$ and 
$\sigma_{\Delta_{\log M_*}}$.
There is a noticeable correlation with $g-r$ and $r-z$ color, which also
appears in the $M_*-{\rm SFR}$ plane. 
However, the effect of this correlation is small $< 0.1$ dex, especially
considering the ${\sim}0.1$ dex accuracy of our inferred posterior on 
$\log M_*$. \\

% log SFR 
\noindent \underline{\emph{Inferred $\log\avgsfr$}}: 
We infer accurate $\log\avgsfr$ for galaxies with $\log\avgsfr > -1$ dex with
${\sim} 0.1$ dex precision. 
Below this limit, we significantly overestimate $\log \avgsfr$, consistent with
the bias in Figure~\ref{fig:prop_inf}. 
In fact, we find a $\log \avgsfr \sim -1$ dex lower bound for the inferred
$\log \avgsfr$. 
At $\log\avgsfr < -1$ dex, we also find that the constraints are significantly
broader with $\sigma_{\Delta_{\log M_*}} \sim 0.25 - 0.3$ dex.
Comparing $\mu_{\Delta_{\theta}}$ and $\sigma_{\Delta_{\theta}}$ from
spectrophotometry versus from only photometry, we confirm that including
spectra significantly improves the accuracy and tightens the $\log\avgsfr$
constraints.
For $\avgsfr$ below $\log\avgsfr < -1$ dex, the bias is reduced by ${\sim}1$
dex --- an order of magnitude. 

We find no significant correlation between the accuracy and precision of
$\avgsfr$ with spectral or photometric SNR.
Meanwhile, there is a more significant color dependence where we overestimate
$\log\avgsfr$ by $\mu_{\Delta_{\log\avgsfr}}>0.5$ dex for the reddest galaxies
($g-r > 1.5$ and $r-z> 0.6$).
The constraints for these galaxies are also significantly less precise:
$\sigma_{\Delta_{\log\avgsfr}} \sim 0.5$ dex. 
The bias is also apparent in Figure~\ref{fig:etas_msfr}. 
There is significant bias in the inferrd for quiescent galaxies where we
overestimate $\avgsfr$. 
$\avgsfr$ is also slightly underestimated for the most massive ($M_* >
10^{11}M_\odot$) star-forming galaxies. 
These biases are consequences of our SPS model priors.
$\avgsfr$ is a derived quantity; hence, the uniform priors we impose on SPS
parameters induce non-uniform priors on them.
Our SPS model imposes a prior on $\log \overline{\rm SSFR}_{\rm 1 Gyr}$
that is skewed and peaks at ${\sim}-10.4$ dex (Appendix~\ref{sec:model_priors}, Figure~\ref{fig:model_priors}). 
Consequently, the posterior overestimates $\avgsfr$ at low $\avgsfr$ (red,
quiescent galaxies) and underestimates $\avgsfr$ at the highest $\avgsfr$. \\

% log Z_MW  
\noindent \underline{\emph{Inferred $\log\zmw$}}:  
In Figure~\ref{fig:etas}, we find that we overestimate $\log\zmw$ by $\sim
0.2$ dex below $\log\zmw < -1.8$ dex and may underestimate $\log\zmw$ at the
highest $\zmw$.
$\sigma_{\Delta_\theta} \sim 0.15$ dex is uniform throughout the $\zmw$ range.
Similar to $\avgsfr$, the bias in inferred $\zmw$ is a consequence of our SPS
model priors. 
The prior skews the $\log\zmw$ constraints towards the peak of the prior 
($\log\zmw\sim-1.5$). 
Figure~\ref{fig:etas}, also demonstrates that including DESI spectra 
substantially improves the accuracy $\log\zmw$ constraints. 
Adding the DESI spectra to the likelihood reduces the relative contribution of
the prior on the posterior. 
Hence DESI spectra plays a pivotal role reducing the impact of the SPS model
priors. 
The constraining power of DESI spectra is further demonstrated in
Figure~\ref{fig:eta_photo}. 
$\zmw$ is significantly overestimated at $r_{\rm fiber} > 20$, where galaxies
have low spectral SNR. 
Both the color and $M_*-{\rm SFR}$ dependences of $\mu_{\Delta_\theta}$ for
$\zmw$ are a consequence of this spectral SNR dependence. 
If we exclude galaxies with low spectral SNR, we infer $\log\zmw$ with
$\mu_{\Delta_\theta}<0.15$ dex and $\sigma_{\Delta_\theta}\sim0.1$ (black
dot-dashed in Figure~\ref{fig:etas}). 

% t_age, MW 



%For $\avgsfr$, Figure~\ref{fig:etas_msfr} helps further characterize the effect
%of the SPS model prior on our constraints. 
%We expectedly find a significant bias for quiescent galaxies since our
%$\log\avgsfr$ constraints have a lower bound of $-1$ dex (Figure~\ref{fig:etas}).
%We also find that $\avgsfr$ is slightly underestimated for the most massive
%($M_* > 10^{11}M_\odot$) star-forming galaxies. 
%For the rest of the galaxies, there is no significant bias in the constraints. 
%In terms of precision, we find no significant $M_*-{\rm SFR}$ dependence in our
%$\log\avgsfr$ constraints. 


%The biases in $\avgsfr$, $Z_{\rm MW}$, $\tage$ are all a consequence of our SPS
%model priors (Appendix~\ref{sec:model_priors}). 
%$\avgsfr$, $Z_{\rm MW}$, and $\tage$ are derived quantities; hence, the uniform
%priors we impose on SPS parameters induce non-uniform priors on them.
%For our SPS model, we impose a prior on $\log \overline{\rm SSFR}_{\rm 1 Gyr}$
%that is skewed and peaks at ${\sim}-10.4$ dex (Figure~\ref{fig:model_priors}). 
%Consequently, the posterior overestimates $\avgsfr$ at low $\avgsfr$.
%The SPS model also imposes skewed priors on $\log Z_{\rm MW}$ and $\tage$,
%which causes to overestimate $\log Z_{\rm MW}$ for low $Z_{\rm MW}$ galaxies
%and underestimate $\tage$ for galaxies with older stellar population. 

We confirm that including spectra significantly tightens the constraints on all
of the galaxy properties. 
\todo{  % update with final calculations
    For instance, $\sigma_{\Delta_{\theta}}$ for $M_*$ improves from X to X dex
    once spectra is included.
} 
In addition to the increased precision, including spectra significantly
reduces the bias in the inferred galaxy properties. 
The overall bias on $Z_{\rm MW}$ is also reduced by $\sim0.3$ dex. 
%Overall, Figure~\ref{fig:etas} illustrates that we derive overall unbiased constraints on galaxy properties, except for low $\avgsfr$ galaxies, and that including spectra in the SED modeling substantially improves the inferred galaxy properties. 



%Again, the imprecision of the $\avgsfr$ constrants is due to model priors (see Section~\ref{sec:discuss} and Appendix~\ref{sec:model_priors}}. 

For $\tage$, we derive unbiased and precise constraints out to $\tage < 8$ Gyr. 
For galaxies with older stellar populations, the log-spaced $\tlb$ binning in
our SPS model (Section~\ref{sec:sps}) expectedly results in underestimated  
$\tage$ constraints with larger uncertainties 
($\sigma_{\Delta_{\tage}} \gtrsim 1$ Gyr). 
Lastly, for $\tauism$ we find a significant $\tauism$ dependence bias in both
accuracy and precision. 
The inferred values increasingly underestimate $\tauism$ with less precise
constraints for greater $\tauism$.

The biases in $\avgsfr$, $Z_{\rm MW}$, $\tage$ are all a consequence of our SPS
model priors (Appendix~\ref{sec:model_priors}). 
$\avgsfr$, $Z_{\rm MW}$, and $\tage$ are derived quantities; hence, the uniform
priors we impose on SPS parameters induce non-uniform priors on them.
For our SPS model, we impose a prior on $\log \overline{\rm SSFR}_{\rm 1 Gyr}$
that is skewed and peaks at ${\sim}-10.4$ dex (Figure~\ref{fig:model_priors}). 
Consequently, the posterior overestimates $\avgsfr$ at low $\avgsfr$.
The SPS model also imposes skewed priors on $\log Z_{\rm MW}$ and $\tage$,
which causes to overestimate $\log Z_{\rm MW}$ for low $Z_{\rm MW}$ galaxies
and underestimate $\tage$ for galaxies with older stellar population. 

The bias in our $\tauism$ constraint is primarily due to intentially included
discrepancies between the SPS model and the mock observations included. 
First, as we stated earlier, we use a dust prescription with a different
attenuation curve in the SPS model than in the forward model. 
This already places a limit on how accurately we can derive $\tauism$; however,
we introduced this discrepancy since we do not know the ``true'' attenuation
curve of observed galaxies in practice. 
Another reason for the biased $\tauism$ constraints is that we only attenuate
the stellar emission in the disk component of the simulated galaxies and not
the bulge component (Section~\ref{sec:sed}).
The true $\tauism$ is the optical depth for the disk componenet while our
$\tauism$ constraints correspond to the optical depth of dust attenuation
for the entire galaxies, a quantity that will be lower than the true $\tauism$
depending on how much the bulge contributes to the SED. 


We also find a significant color dependence in the $Z_{\rm MW}$ constraints.
This is primarily driven by the color dependence in the spectral SNR ($r_{\rm
fiber}$).  
If we exclude galaxies with $r_{\rm fiber} > 20$, we find no significant color
dependence and $\mu_{\Delta_{\log Z_{\rm MW}}} < 0.1$ dex.
For $\tage$, there is no significant color dependence except at $r - z > 0.6$,
where $\tage$ is underestimated. 
However, this $r - z$ dependence is simply a consequence of $\tage$ being
underestimated for $\tage > 8\,{\rm Gyr}$, which was due to the log $\tlb$
binning (Figure~\ref{fig:etas}); the simulated galaxies with $r - z > 0.6$ in
our sample have overall older stellar populations.
Lastly, we find no significant color dependence on $\tauism$. 

%As we mention earlier, the accuracy of our $Z_{\rm MW}$ constraint depends significantly on the spectral SNR ($r_{\rm fiber}$).
%meanwhile SFR is much sharper so it's more affected by the true galaxy properties This can then transition into the accuracy and precision as a function of galaxy type. 

Starting with $M_*$, we confirm that there is no significant bias in the
inferred $\log M_*$ in the top panel. 
We also find that the precision of the constraints are equal for both
star-forming and quiescent galaxies. 

For $\log Z_{\rm MW}$, star-forming galaxies on the so-called ``star-forming
sequence'' (SFS) have higher $\mu_{\Delta_{\log Z_{\rm MW}}}$.
However, this correlation is primarily driven by the $r_{\rm fiber}$ dependence. 
Without $r_{\rm fiber} > 20$ galaxies, we find little correlation between
$\mu_{\Delta_{\log Z_{\rm MW}\theta}}$ and the $M_*-{\rm SFR}$ plane. 
We find uniform $\sigma_{\Delta_{\log Z_{\rm MW}}}$ for all types of galaxies. 
Next, for $\tage$, we do not find a clear $M_*-{\rm SFR}$ dependence. 
However, $|\mu_{\Delta_{\tage}}|$ is larger and constraints are significantly
less precise for quiescent galaxies. 
This, again, is a consequence of our log-spaced $\tlb$ binning. 
Lastly, we find unbiased and precise $\tauism$ constraints for all galaxies
except star-forming galaxies above $M_* > 10^{11}M_\odot$ where we
underestimate $\tauism$. 
This is because the massive star-forming galaxies are the only ones with
$\tauism > 1$. 
We again emphasize that a large portion of this bias is due to the discrepancy
in the dust prescriptions of the SPS model versus the mock observations.  

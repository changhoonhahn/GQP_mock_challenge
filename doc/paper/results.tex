\begin{figure}
\begin{center}
\includegraphics[width=0.95\textwidth]{figs/inferred_props.pdf}
\caption{
    Comparison between the true galaxy properties, $\theta_{\rm true}$, and
    those inferred from SED modeling of mock observations, $\hat{\theta}$. 
    From the left to right columns, we compare $\log M_*$, $\log \avgsfr$, 
    $\log Z_{\rm MW}$, and $\tau_{\rm ISM}$. 
    The inferred galaxy properties are derived from SED modeling of mock
    spectra (top), photometry (middle), and spectra + photometry (bottom). 
    For each simulated galaxy, we represent the marginalized posterior of
    $\theta$ with a violin plot.  
    %We derive unbiased and precise $\log M_*$
}
\label{fig:prop_inf}
\end{center}
\end{figure}

\section{Results} \label{sec:results}
%\subsection{Inferred Galaxy Properties}
The goal of this work is to demonstrate the precision and accuracy of inferred
galaxy properties for PROVABGS. 
We apply our SED modeling to the mock observables of \todo{100} \lgal~galaxies.
From the posterior distributions of the SPS parameters, we derive the following
physical galaxy properties: stellar mass ($M_*$), SFR averaged over 1 Gyr
($\avgsfr$), mass-weighted stellar metallicity ($Z_{\rm MW}$), and diffuse-dust
optical depth ($\tau_{\rm ISM}$).
$M_*$ and $\tau_{\rm ISM}$ are both SPS model parameters. 
We derive $\avgsfr$ and $Z_{\rm MW}$ as 
\begin{equation}
    \avgsfr = \frac{\int\limits_{t_{\rm age} - {\rm 1 Gyr}}^{t_{\rm age}}{\rm
    SFH}(t)\,{\rm d}t}{{\rm 1 Gyr}} \quad{\rm and}\quad
    Z_{\rm MW} = \frac{\int\limits_0^{t_{\rm age}}{\rm SFH}(t)\,{\rm
    ZH}(t)\,{\rm d}t}{M_*}.
\end{equation} 

In Figure~\ref{fig:prop_inf}, we compare the galaxy properties inferred from
SED modeling the mock observations, $\hat{\theta}$ to the true (input) galaxy
properties, $\theta_{\rm true}$ of the simulated galaxies.
In each column, we compare $\log M_*$, $\log \avgsfr$, $\log Z_{\rm MW}$, and
$\tau_{\rm ISM}$ from left to right. 
The inferred properties in the top, middle, and bottom rows are derived from
SED modeling of spectra, photometry, and spectra + photometry, respectively.
In each panel, we plot $\hat{\theta}$ using a violin plot, where the width
of the marker represents the marginalized posterior distribution of $\theta$. 
We note that in the comparison with SED modeling spectra only, we do not
include $f_{\rm fiber}$. 
Therefore, the true stellar mass in this case corresponds to $f_{\rm fiber}
\times M_*$. 
\emph{Overall, the comparison demonstrates that we can robustly infer galaxy
properties using the {\sc PROVABGS} SED modeling}. 

In more detail, we find that we infer unbiased and precise constraints on
$M_*$ throughout the entire $M_*$ range. 
%For spectra+photometry SED modeling, the posteriors have $\sigma_{M_*}\sim0.06$ dex.
We also infer robust $\avgsfr$ for $\log \avgsfr > -2$ dex; below this limit,
however, the inferred $\avgsfr$ are significantly less precise and
overestimated the true $\avgsfr$. 
This bias at low $\avgsfr$ is caused by model priors, which we discuss in
further detail later in \todo{Section~\ref{sec:discuss} and
Appendix~\ref{sec:model_priors}}. 
$Z_{\rm MW}$ is not precisely constrained ($\sigma_{\log Z_{\rm MW}} \sim 0.1$
dex); however, there is no strong bias in the constraints. 
Lastly, $\tau_{\rm ISM}$ is accurately inferred for simulated galaxies with
low $\tau_{\rm ISM}$ but the accuracy declines significantly for high 
$\tau_{\rm ISM}$ galaxies. 

The fact that we infer overall unbiased constrainted on galaxy properties for
the mock observations is especially encouraging because there are significant
differences in the forward model used to generate the observations and the SPS
model used in the SED modeling. 
First, the SFH in the forward model is taken directly from \lgal~simulation
outputs. 
Meanwhile, the SFH parameterization in the SPS model is based on NMF bases fit
on SFHs of IllustrisTNG galaxies.
Even though the there are significant differences between the SFHs of \lgal~and
IllustrisTNG, the NMF SFH prescription is able to accurately reproduce recent
SFH.
Furthermore, in the forward model we construct the SED of the bulge and disk
components fo the simulated galaxies separately. 
There are also differences between the dust prescriptions in the forward model
versus the SPS model.
Nevertheless, we recover overall unbiased $\avgsfr$ and $Z_{\rm MW}$. 

In Figure~\ref{fig:prop_inf}, we can also compare spectra, photometry, versus
SP.


In order to quantify the precision and accuracy of the inferred galaxy
properties for our simulated galaxy population, we begin by assuming that the
discrepancy between the inferred and true parameters for each galaxy 
$\Delta_{\theta,i}$) 
\begin{equation}
    \theta^{\rm inf}_i = \theta^{\rm true}_i + \Delta_{\theta,i}
\end{equation}
where $\Delta_{\theta,i}$ is sampled from a Gaussian distribution
\begin{equation}
    \Delta_{\theta,i} \sim \mathcal{N}(\mu_{\Delta_{\theta}}, \sigma_{\Delta_{\theta}}).
\end{equation}
This Gaussian distribution is described by population hyperparameters $\mu_{\Delta_{\theta}}$ and 
$\sigma_{\Delta_{\theta}}$, the mean and standard deviation, which quantify the accuracy and 
precision of the inferred physical properties for the population. 

Given the photomety and spectrum of our galaxies, $\{{\bfi D}_i\}$, we can get the posteriors
for these population hyperparameters $\eta_\Delta = \{\mu_{\Delta_{\theta}},
\sigma_{\Delta_{\theta}}\}$ 
using a hierarchical Bayesian framework~\citep{hogg2010a}: 
\begin{align}
p(\eta_\Delta \given \{{\bfi D_i}\}) 
    =&~\frac{p(\eta_\Delta)~p( \{{\bfi D_i}\} \given \theta_{\Delta})}{p(\{{\bfi D_i}\})}\\
    =&~\frac{p(\eta_\Delta)}{p(\{{\bfi D_i}\})}\int p(\{{\bfi D_i}\} \given \{\theta_i\})~p(\{\theta_i\} \given \eta_\Delta)~{\rm d}\{\theta_i\}.
\end{align} 
Naively the posteriors for each of the galaxies are not correlated, so we can factorize the expression above
\begin{align}
p(\eta_\Delta \given \{{\bfi D_i}\}) 
    =&~\frac{p(\eta_\Delta)}{p(\{{\bfi D_i}\})}\prod\limits_{i=1}^N\int p({\bfi D_i} \given \theta_i)~p(\theta_i \given \theta_\Delta)~{\rm d}\theta_i\\
    =&~\frac{p(\eta_\Delta)}{p(\{{\bfi D_i}\})}\prod\limits_{i=1}^N\int
    \frac{p(\theta_i \given {\bfi D_i})~p({\bfi D_i})}{p(\theta_i)}~p(\theta_i
    \given \eta_\Delta)~{\rm d}\theta_i\\
    =&~p(\eta_\Delta)\prod\limits_{i=1}^N\int \frac{p(\theta_i \given {\bfi
    D_i})~p(\theta_i \given \eta_\Delta)}{p(\theta_i)}~{\rm d}\theta_i.
\end{align} 
$p(\theta_i \given {\bfi D_i})$ is the posterior for galaxy $i$. Hence, the
integral can be which means the integral can be estimated using the MCMC sample
from the posterior
\begin{align}
p(\eta_\Delta \given \{{\bfi D_i}\}) 
    =&~p(\eta_\Delta)\prod\limits_{i=1}^N\frac{1}{S_i}\sum\limits_{j=1}^{S_i}
    \frac{p(\theta_{i,j} \given \eta_\Delta)}{p(\theta_{i,j})}.
\end{align} 
$S_i$ is the number of MCMC samples and $\theta_{i,j}$ is the $j^{\rm th}$
sample of galaxy $i$. We present the maximum a posteriori (MAP) estimates of
$\eta_\Delta$ for $\log~M_*$ and $\log~{\rm SFR}$ in
Figure~\ref{fig:specphoto}. 


$\eta_\Delta$ as a function of SNR/mag/colour. 

\begin{figure}
\begin{center}
\includegraphics[width=0.95\textwidth]{figs/etas.pdf} 
\caption{The discrepancies between the inferred and input/``true'' $M_*$s (left) and SFRs 
(right) for our {\sc LGal} galaxies. In blue, we infer $M_*$s and SFRs using only photometry;
in orange, we infer $M_*$s and SFRs by jointly fitting both photometry and spectroscopy. 
{\em Jointly fitting spectroscopy and photometry improves constraints on galaxy properties.}
}
\label{fig:specphoto}
\end{center}
\end{figure}

\begin{figure}
\begin{center}
    \includegraphics[width=0.95\textwidth]{figs/etas_photo.pdf}
    \caption{do we want to test bias against other properties? (e.g. obs condition?)} 
    \label{fig:systematics}
\end{center}
\end{figure}

%\begin{table}
%\caption{$\theta_\Delta$ $\theta_{\rm inf}$ - $\theta_{\rm true}$ and uncertainties for different sets of data fitted with ifsps} 
%\begin{center} 
%\begin{tabular}{ccccc} \toprule
%set & photometry & spectroscopy & specphot \\
%$\Delta M_{tot}$ & 0.13 & 0.11 & 0.09\\
%$M_{err}$ & 0.10 & 0.08 & 0.07 \\
%$\Delta$ Age & 4.05 & 3.84 & 4.33\\
%$Age_{err}$ & 1.83 & 2.37 & 2.03\\
%$\Delta$ Z & 0.0546 & 0.0126 & 0.0063 \\
%$Z_{err}$ & 0.0291 & 0.0203 & 0.0050\\
%\hline 
%\hline            
%\end{tabular} \label{tab:setups}
%\end{center}
%\end{table}

%\begin{figure}
%\begin{center}
%\includegraphics[width=\textwidth]{figs/mini_mocha_cigale_noise_CIGALEA.png} 
%\includegraphics[width=\textwidth]{figs/mini_mocha_cigale_noise_CIGALEB.png} 
%\includegraphics[width=\textwidth]{figs/mini_mocha_cigale_noise_CIGALEC.png}
%\includegraphics[width=\textwidth]{figs/mini_mocha_cigale_noise_CIGALED.png}
%\caption{The properties inferred from CIGALE photometry fit as a function of true properties. Configuration CIGALE A, B, C, and D 
%}
%\label{fig:photo_cigale}
%\end{center}
%\end{figure}

%\begin{figure}
%\begin{center}
%\includegraphics[width=\textwidth]{figs/mini_mocha_ifsps_specphotofit_vanilla_noise_bgs0_legacy_delta.pdf} 
%\caption{delta(galaxy properties) as a function of $M_{tot}$, r mag and colors for ifsps(spectrophotometry) and CIGALE (photometry, CIGALE D). 
%}
%\label{fig:photo_cigaleALL}
%\end{center}
%\end{figure}

discussion: 
\begin{itemize}
    \item science applicatoins 
    \item comparison to other methods 
    \item different observing condition doesn't impact our results? 
\end{itemize}


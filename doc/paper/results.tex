\begin{figure}
\begin{center}
\includegraphics[width=0.9\textwidth]{figs/inferred_props.pdf}
\caption{
    Comparison between the true galaxy properties, $\theta_{\rm true}$, and
    those inferred from SED modeling of mock observations, $\hat{\theta}$. 
    From the left to right columns, we compare $\log M_*$, $\log \avgsfr$, 
    $\log Z_{\rm MW}$, $\tage$ and $\tauism$. 
    The inferred galaxy properties are derived from SED modeling of mock
    spectra (top), photometry (middle), and spectrophotometry (bottom). 
    For each simulated galaxy, we represent the marginalized posterior of
    $\theta$ with a violin plot.  %We derive unbiased and precise $\log M_*$
    \emph{The posteriors demonstrate that, overall, we can derive accurate and 
    precise constraints on certain galaxy properties from joint SED modeling of
    DESI photometry and spectra.}
    } \label{fig:prop_inf}
\end{center}
\end{figure}

\section{Results} \label{sec:results}
%\subsection{Inferred Galaxy Properties}
The goal of this work is to demonstrate the precision and accuracy of inferred
galaxy properties for PROVABGS. 
We apply our SED modeling to the mock observables of 2,123 \lgal~galaxies.
From the posterior distributions of the SPS parameters, we derive the following
physical galaxy properties: stellar mass ($M_*$), SFR averaged over 1 Gyr
($\avgsfr$), mass-weighted stellar metallicity ($Z_{\rm MW}$), mass-weighted
stellar age ($\tage$), and diffuse-dust optical depth ($\tau_{\rm ISM}$).
$M_*$ and $\tau_{\rm ISM}$ are SPS model parameters, while $\avgsfr$, 
$Z_{\rm MW}$, and $\tage$ are derived as 
\begin{equation} \label{eq:prop_eqs}
    \avgsfr = \frac{\int\limits_{t_{\rm age} - {\rm 1 Gyr}}^{t_{\rm age}}{\rm
    SFH}(t)\,{\rm d}t}{{\rm 1 Gyr}}, \quad
    Z_{\rm MW} = \frac{\int\limits_0^{t_{\rm age}}{\rm SFH}(t)\,{\rm
    ZH}(t)\,{\rm d}t}{M_*}, \quad{\rm and}\quad
    \tage = \frac{\int\limits_0^{t_{\rm age}}{\rm SFH}(t)\,t\,{\rm d}t}{M_*}.
\end{equation} 

In Figure~\ref{fig:prop_inf}, we compare the galaxy properties inferred from
SED modeling the mock observations, $\hat{\theta}$, to the true (input) galaxy
properties, $\theta_{\rm true}$, of the simulated galaxies.
From left to right, we compare $\log M_*$, $\log \avgsfr$, $\log Z_{\rm MW}$,
$\tage$, and $\tauism$ in each column.  
The inferred properties in the top, middle, and bottom rows are derived from
SED modeling of spectra, photometry, and spectrophotometry, respectively.
In each panel, we plot $\hat{\theta}$ using a violin plot, where the width
of the marker represents the marginalized posterior distribution of $\theta$. 
We note that in the comparison with SED modeling spectra only, we do not
include $f_{\rm fiber}$ so the true stellar mass in this case corresponds to
$f_{\rm fiber} \times M_*$. 
The comparison demonstrates that \emph{overall we robustly infer galaxy
properties using the {\sc PROVABGS} SED modeling}. 

In more detail, we find that we infer unbiased and precise constraints on
$M_*$ throughout the entire $M_*$ range. 
%For spectra+photometry SED modeling, the posteriors have $\sigma_{M_*}\sim0.06$ dex.
We also infer robust $\avgsfr$ above $\log \avgsfr > -1$ dex; below this limit,
however, the inferred $\avgsfr$ are significantly less precise and
overestimate the true $\avgsfr$. 
This bias at low $\avgsfr$ is caused by model priors, which we discuss in
further detail later in Section~\ref{sec:discuss} and
Appendix~\ref{sec:model_priors}. 
Both $Z_{\rm MW}$ and $\tage$ are not precisely constrained; however, we do not
find clear biases in the constraints. 
For $\tage$, the posteriors are less precise for galaxies with older stellar
populations and they reveal the log-spaced $\tlb$ binning used in our
SPS model for $\tage > 6$ Gyr.
Lastly, $\tauism$ is overall accurately inferred for galaxies with low
$\tauism$ but appears to be underestimated for high $\tauism$.

The overall constraints on galaxy properties for the mock observations is
especially encouraging due to the significant differences in the forward
model used to generate the observations and the SPS model used in the SED
modeling. 
First, the SFHs and ZHs in the mock observations are taken directly from
\lgal~simulation outputs while the SFH and ZH parameterization in the SPS model
is based on NMF bases fit to Illustris galaxies.
Second, in the forward model, we construct the SED of the bulge and disk
components of the simulated galaxies separately: the components have separate
SFHs and ZHs. 
The SPS model treats all galaxies as having one component. 
Lastly, we use different dust prescriptions: \cite{mathis1983} dust
attenuation curve in the forward model and \cite{kriek2013} dust attenuation
curve in the SPS model. 
Despite these significant differences, our constraints on certain galaxy
properties are unbiased and precise. 

Figure~\ref{fig:prop_inf}, also highlights the advantages of jointly modeling
spectra and photometry. 
Comparing the constraints from spectrophotometry (bottom) versus photometry
alone (middle), we find that including spectra significantly tightens the
constraints for all properties. 
In addition, including spectra also appears to reduce biases of the
constraints. 
For instance, with only photometry, we derive significantly more biased
$\avgsfr$ constraints.
This is due to the limited constraining power of photometry, which allows the
posteriors to be dominated by model priors. 
Adding spectra, significantly increases the contribution of the likelihood and
ameliorates this effect. 
%Therefore, joint SED modeling of spectra and photometry 

Beyond qualitative comparisons of the posterior, we want to quantify the
precision and accuracy of the inferred galaxy properties. 
Let $\Delta_{\theta,i}$ be the discrepancy between the inferred and true
parameters for each galaxy: 
$\Delta_{\theta,i} = \hat{\theta}_i - \theta^{\rm true}_i$.
Then, if we assume that $\Delta_{\theta,i}$ are sampled from a Gaussian
distribution,
\begin{equation} \label{eq:eta_gauss}
    \Delta_{\theta,i} \sim \mathcal{N}(\mu_{\Delta_{\theta}}, \sigma_{\Delta_{\theta}}),
\end{equation}
the mean ($\mu_{\Delta_{\theta}}$) and standard deviation
($\sigma_{\Delta_{\theta}}$) of the distribution are population hyperparameters
that represent the accuracy and precision of the inferred posteriors for the
galaxy population. 
We can infer $\mu_{\Delta_{\theta}}$ and $\sigma_{\Delta_{\theta}}$ using a
hierarchical Bayesian framework~\citep[\emph{e.g.}][]{hogg2010,
foreman-mackey2014, baronchelli2020}.

Let $\{{\bfi X}_i\}$ represent the photometry or spectrum of a galaxy
population and $\eta_\Delta = \{\mu_{\Delta_{\theta}},
\sigma_{\Delta_{\theta}}\}$ represent the population hyperparameters.
Our goal is to constrain $\eta_\Delta$ from $\{{\bfi X}_i\}$ --- \emph{i.e.}
to infer $p(\eta_\Delta \given \{{\bfi X}_i\})$.
We expand 
\begin{align}\label{eq:popinf}
p(\eta_\Delta \given \{{\bfi X_i}\}) 
    =&~\frac{p(\eta_\Delta)~p( \{{\bfi X_i}\} \given \eta_\Delta)}{p(\{{\bfi X_i}\})}\\
    =&~\frac{p(\eta_\Delta)}{p(\{{\bfi X_i}\})}\int p(\{{\bfi X_i}\} \given \{\theta_i\})~p(\{\theta_i\} \given \eta_\Delta)~{\rm d}\{\theta_i\}.
\intertext{
    $\theta_i$ is the SPS parameters for galaxy $i$ and $p(\{{\bfi X_i}\}
    \given \{\theta_i\})$ is likelihood of the set of observations $\{{\bfi
    X_i}\}$ given the set of $\{\theta_i\}$. 
    Since the likelihoods for each of the $N$ galaxies, $p(\bfi X_i \given
    \theta_i)$, are not correlated, we can factorize and write the expression
    above as 
}
    =&~\frac{p(\eta_\Delta)}{p(\{{\bfi X_i}\})}\prod\limits_{i=1}^N\int p({\bfi X_i} \given \theta_i)~p(\theta_i \given \eta_\Delta)~{\rm d}\theta_i\\
    =&~\frac{p(\eta_\Delta)}{p(\{{\bfi X_i}\})}\prod\limits_{i=1}^N\int
    \frac{p(\theta_i \given {\bfi X_i})~p({\bfi X_i})}{p(\theta_i)}~p(\theta_i
    \given \eta_\Delta)~{\rm d}\theta_i\\
    =&~p(\eta_\Delta)\prod\limits_{i=1}^N\int \frac{p(\theta_i \given {\bfi
    X_i})~p(\theta_i \given \eta_\Delta)}{p(\theta_i)}~{\rm d}\theta_i. 
\intertext{
    $p(\theta_i \given {\bfi X_i})$ is the posterior for an individual galaxy,
    so the integral can be estimated using the Monte Carlo samples from the
    posterior: 
}
    \approx&~p(\eta_\Delta)\prod\limits_{i=1}^N\frac{1}{S_i}\sum\limits_{j=1}^{S_i}
    \frac{p(\theta_{i,j} \given \eta_\Delta)}{p(\theta_{i,j})}.
    \label{eq:popinf2}
\end{align} 
$S_i$ is the number of posterior samples and $\theta_{i,j}$ is the $j^{\rm th}$
sample of galaxy $i$.
$p(\theta_{i,j} \given \eta_\Delta) = p(\Delta_{\theta,i,j} \given
\eta_\Delta)$ is a Gaussian distribution and, hence, easy to evaluate. 
$p(\theta_{i,j}) = 1$ since we use uninformative and Dirichlet priors
(Table~\ref{tab:params}). 
Finally, we derive the maximum a posteriori (MAP) value of $\eta_\Delta$ by
maximizing the $p(\eta_\Delta \given \{{\bfi X_i}\})$ posterior distribution.
This type of population inference is a major advantage of inferring full
posteriors distributions of the galaxy properties.
We discuss the derivation and interpretation of the hyperparameters in more
detail in Appendix~\ref{sec:hyper}.

\begin{figure}
\begin{center}
    \includegraphics[width=0.85\textwidth]{figs/etas_v2.pdf}
    \caption{
        The accuracy and precision of galaxy property posteriors from our
        joint SED modeling of spectrophotometry, quantified using population
        hyperparameters $\eta_\Delta = \{\mu_{\Delta_{\theta}},
        \sigma_{\Delta_{\theta}}\}$, as a function of true galaxy property
        (green). 
        We plot $\theta_{\rm true} + \mu_{\Delta_{\theta}}$ in solid line and
        represent $\sigma_{\Delta_{\theta}}$ with the shaded region.
        We include $\eta_\Delta$ for SED modeling of photometry alone (orange)
        for comparison. 
        Including DESI spectra significantly improves both the accuracy and
        precision of the inferred galaxy properties. 
        $\log\avgsfr$, $\log Z_{\rm MW}$, and $\tage$ constraints are
        significantly impacted by  priors imposed by the SPS model
        (Appendix~\ref{sec:model_priors}).
        Discrepancies in the dust prescriptions between our SPS model and the
        mock observations drive the bias in $\tauism$.
        Nevertheless, \emph{we accurately and precisely infer: $\log M_*$ for
        all $M_*$, $\log\avgsfr$ above $\log\avgsfr > -1\,{\rm dex}$, and 
        $\tage$ below $8\,{\rm Gyr}$.}
        } \label{fig:etas}
\end{center}
\end{figure}

In Figure~\ref{fig:etas}, we present the accuracy ($\mu_{\Delta_{\theta}}$) and
precision ($\sigma_{\Delta_{\theta}}$) of our joint SED modeling of spectra and
photometry (green) as a function of true galaxy property. 
$\mu_{\Delta_{\theta}}$ (solid) and $\sigma_{\Delta_{\theta}}$ (shaded region)
are the MAP values of $p(\eta_\Delta \given \{{\bfi X_i}\})$ posterior. 
In each panel, we derive $p(\eta_\Delta \given \{{\bfi X_i}\})$ for 
$\log M_*$, $\avgsfr$, $\log Z_{\rm MW}$, $\tage$, and $\tauism$ in bins of
widths 0.2 dex, 0.5 dex, 0.05 dex, 0.5 Gyr, and 0.1, respectively. 
We only include bins with more than ten galaxies. 
For comparison, we include $\eta_\Delta$ for SED modeling of photometry alone
(orange).
We also include $\eta_\Delta$ for $\log\zmw$ of galaxies with $r_{\rm fiber} >
20$ (black dot-dashed) and $\eta_\Delta$ for $\tauism$ of galaxies
without bulges (black dotted), which we discuss later. 

\begin{figure}
\begin{center}
    \includegraphics[width=0.95\textwidth]{figs/etas_photo.pdf}
    \caption{
        Accuracy and precision of the galaxy properties inferred from joint SED
        modeling of spectrophotometry as a function of $r_{\rm fiber}$, $r$,
        $g-r$, and $r-z$.
        $r_{\rm fiber}$ and $r$ magnitudes are proxies for spectral and
        photometric SNR. 
        From the top to bottom rows, we present $\eta_\Delta$ for $\log M_*$,
        $\log\avgsfr$, $\log Z_{\rm MW}$, $t_{\rm age, MW}$ and 
        $\tau_{\rm ISM}$.
        We find a significant dependence on spectral SNR in the inferred 
        $\log \zmw$. 
        When the spectral SNR is low ($r_{\rm fiber} > 20$), the prior on 
        $\log \zmw$ imposed by the SPS model dominate the posterior and
        cause overestimates of $Z_{\rm MW}$. 
        We find a significant color dependence on $\log\avgsfr$, $\log Z_{\rm
        MW}$, and $\tage$. 
        For $\log\zmw$ and $\tage$, the dependence is driven by underlying
        correlations with spectral SNR and true $\tage$. 
        Meanwhile, $\log\avgsfr$ is overestimated for the reddest galaxies with
        $r - z > 0.6$, which correspond to quiescent galaxies with $\log\avgsfr
        < -1$ dex. 
        Otherwise we find no significant dependence on SNR or optical color. 
    }    
    \label{fig:eta_photo}
\end{center}
\end{figure}

In Figure~\ref{fig:eta_photo}, we examine how the accuracy and precision of
our galaxy parameter constraints are impacted by signal-to-noise ratio (SNR) or
photometric color. 
We present $\eta_\Delta$ of our joint SED modeling of spectra and photometry as
a function of $r_{\rm fiber}$, $r$, $g-r$, and $r-z$. 
$r_{\rm fiber}$ and $r$ magnitudes serve as proxies of the SNR for the spectra
and photometry, respectively. 
In each row, we plot $\eta_\Delta$ for a different galaxy property: $\log M_*$,
$\avgsfr$, $\log Z_{\rm MW}$, $\tage$ and $\tauism$ (from top to bottom).

\begin{figure}
\begin{center}
    \includegraphics[width=\textwidth]{figs/etas_msfr.pdf} 
    \caption{
        Accuracy and precision of the galaxy properties inferred from joint SED
        modeling of spectrophotometry as a function of the galaxies' true $M_*$
        and $\avgsfr$. 
        We present $\mu_{\Delta_{\theta}}$ and $\sigma_{\Delta_{\theta}}$ in
        ($M_*$, $\avgsfr$) bins for $\log M_*$, $\log\avgsfr$, $\log Z_{\rm
        MW}$, $t_{\rm age, MW}$ and $\tau_{\rm ISM}$ in the top and bottom
        panels respectively. 
        $\log M_*$ is accurately and precisely constrained for all types of galaxies. 
        $\log\avgsfr$ is accurately and precisely constrained for all galaxies
        except for quiescent galaxies with $\log\avgsfr > -1$ dex. 
        $\log\zmw$ is overestimated for star-forming galaxies, due to their
        overall lower spectral SNR. 
        $\tage$ is accurately and precisely constrained for star-forming
        galaxies that have overall younger stellar populations. 
        $\tauism$ is accurately and precisely constrained for all galaxies
        except massive star-forming galaxies, which have high true $\tauism$. 
    }\label{fig:etas_msfr}
\end{center}
\end{figure}

Lastly, in Figure~\ref{fig:etas_msfr}, we investigate whether there is any
underlying dependence in the inferred galaxy properties on the  
$M_*$-${\rm SFR}$ plane. 
In the top and bottom panels, we present $\mu_{\Delta_{\theta}}$ and 
$\sigma_{\Delta_{\theta}}$ in $(\log M_*, \log\avgsfr)$ bins for 
$\log M_*$, $\log\avgsfr$, $\log Z_{\rm MW}$, $\tage$ and $\tauism$ (left to
right).
We use $\log M_*$ bins of width 0.225 dex and $\log\avgsfr$ bins of width 
0.25 dex for $\log \avgsfr > 0$ dex and 0.5 dex for $\log \avgsfr < 0$ dex. 
We only present bins with more than 10 galaxies. 
On the $M_*-{\rm SFR}$ plane, we can examine whether the accuracy and precision
of the inferred properties have significant dependencies for galaxy type. 

Based on Figures~\ref{fig:etas}, \ref{fig:eta_photo}, and~\ref{fig:etas_msfr},
we draw the following conclusions on the accuracy and precision of the inferred
posteriors for each galaxy property:\\

% log M*
\noindent \underline{\emph{Inferred $\log M_*$}}: 
Overall, we infer accurate and precise $\log M_*$ from the {\sc PROVABGS} SED
modeling. 
There is no significant dependence in $\mu_{\Delta_{\theta}}$ and
$\sigma_{\Delta_{\theta}}$ with true $\log M_*$ throughout the $M_*$ range. 
We accurately infer the true $M_*$ throughout ${\sim}10^{9}$- $10^{12} M_\odot$
with uniform precision of $\sigma_{\Delta_{\log M_*}}{\sim}0.1$ dex. 
We also find no significant dependence on SNR for $M_*$ --- neither $r_{\rm
fiber}$ nor $r$ magnitudes significantly affect $\mu_{\Delta_{\log M_*}}$ and
$\sigma_{\Delta_{\log M_*}}$.
There is a noticeable correlation with $g-r$ and $r-z$ color, which also
appears in the $M_*-{\rm SFR}$ plane. 
However, this correlation is small compared to the precision of our inferred
posterior on $\log M_*$. 
When we compare the $\eta_\Delta$ from spectrophotometry to $\eta_\Delta$ from
photometry we find that including DESI spectra increases both the accuracy and
precision of the constraints, especially at high $M_* > 10^{11}M_\odot$. \\

% log SFR 
\noindent \underline{\emph{Inferred $\log\avgsfr$}}: 
We infer accurate $\log\avgsfr$ for galaxies with $\log\avgsfr > -1$ dex with
${\sim} 0.1$ dex precision. 
In fact, we find a $\log \avgsfr \sim -1$ dex lower bound for the inferred
$\log \avgsfr$.
Below this limit, we significantly overestimate $\log \avgsfr$, consistent with
the bias in Figure~\ref{fig:prop_inf}, and the constraints are also
significantly broader with $\sigma_{\Delta_{\log M_*}}{\sim}0.25 - 0.3$ dex.
Comparing $\mu_{\Delta_{\theta}}$ and $\sigma_{\Delta_{\theta}}$ from
spectrophotometry versus from only photometry, we confirm that including
spectra significantly improves the accuracy and tightens the $\log\avgsfr$
constraints.
For $\avgsfr$ below $\log\avgsfr < -1$ dex, including spectra reduces the bias
${\sim}1$ dex --- an order of magnitude. 

We find no significant correlation between the accuracy and precision of
$\avgsfr$ with spectral or photometric SNR.
Meanwhile, there is a more significant color dependence where we overestimate
$\log\avgsfr$ by $\mu_{\Delta_{\log\avgsfr}}{>}0.5$ dex for the reddest galaxies
($g-r > 1.5$ and $r-z> 0.6$).
The constraints for these galaxies are also significantly less precise:
$\sigma_{\Delta_{\log\avgsfr}} \sim 0.5$ dex. 
The bias is also apparent in Figure~\ref{fig:etas_msfr}: there is significant
bias in the inferred for quiescent galaxies where we overestimate $\avgsfr$. 
$\avgsfr$ is also slightly underestimated for the most massive ($M_* >
10^{11}M_\odot$) star-forming galaxies. 
These biases are consequences of our SPS model priors.
$\avgsfr$ is a derived quantity; hence, the uninformative priors we impose on
SPS parameters induce non-uniform priors on them.
Our SPS model imposes a prior on $\log \overline{\rm SSFR}_{\rm 1 Gyr}$
that is skewed towards the peaks at $\sim$-10.4 dex
(Appendix~\ref{sec:model_priors}, Figure~\ref{fig:model_prior}). 
Consequently, the posterior overestimates $\avgsfr$ at low $\avgsfr$ (red,
quiescent galaxies) and underestimates $\avgsfr$ at the highest $\avgsfr$. \\

% log Z_MW  
\noindent \underline{\emph{Inferred $\log\zmw$}}:  
Unlike in Figure~\ref{fig:prop_inf}, $\eta_\Delta$ in Figure~\ref{fig:etas}
clearly reveals the accuracy and precision of the posteriors on $\log\zmw$. 
We find that $\mu_{\Delta_{\theta}}$ depends significantly on the true $\zmw$: 
inferred $\log\zmw$ is overestimated by ${\sim}0.2$ dex below $\log\zmw <
-2$ dex and slightly underestimated at the highest $\log\zmw > -1.6$ dex.
$\sigma_{\Delta_\theta} \sim 0.15$ dex is uniform throughout the $\zmw$ range.
Similar to $\avgsfr$, the bias in inferred $\zmw$ is a consequence of our SPS
model priors. 
The prior skews $\log\zmw$ constraints towards the peak of the prior at
$\log\zmw\sim-1.5$. 
Figure~\ref{fig:etas} also includes $\eta_\Delta$ for posteriors derived from
photometry alone (orange), which demonstrates that including DESI spectra
substantially improves the accuracy of the $\log\zmw$ constraints.
Spectra reduces the overall bias on $\zmw$ by $\sim$0.3 dex. 
The improvement comes from the likelihood contribution from DESI spectra
reducing the relative contribution of the prior on the posterior. 

This is also why we find that the posteriors overestimate $\log\zmw$ at 
$r_{\rm fiber} > 20$ in Figure~\ref{fig:eta_photo}.
These correspond to mock observations with low spectral SNR where the
contribution of the likelihood from the spectra is reduced and the prior on
$\log\zmw$ has a larger effect.
The color dependence of $\mu_{\Delta_\theta}$ for $\zmw$ in
Figure~\ref{fig:eta_photo} is also a consequence of this spectral SNR
dependence; so is the $M_*-{\rm SFR}$ dependence (Figure~\ref{fig:etas_msfr}).
If we exclude galaxies with low spectral SNR, both the color and $M_*-{\rm
SFR}$  dependences are substantially reduced.
For only $r_{\rm fiber} < 20$ we infer $\log\zmw$ with
$\mu_{\Delta_\theta}<0.15$ dex and $\sigma_{\Delta_\theta}\sim0.1$
(Figure~\ref{fig:etas}; black dot-dashed). 
The posteriors on $\zmw$ further underscore the constraining power of DESI
spectra. \\

% t_age, MW 
\noindent \underline{\emph{Inferred $\tage$}}:  
Figure~\ref{fig:etas} confirms that we derive unbiased and precise constraints
on $\tage$ out to $\tage < 8$ Gyr. 
Below this limit, we infer $\tage$ with $\sigma_{\Delta_\theta}{\sim}0.5$ Gyr.
For galaxies with older stellar populations above this limit, the log-spaced
$\tlb$ binning in our SPS model (Section~\ref{sec:sps}) expectedly
underestimates $\tage$ constraints and produces larger uncertainties 
($\sigma_{\Delta_{\tage}} \gtrsim 1$ Gyr). 
Meanwhile, we find no significant SNR or color dependence in
Figure~\ref{fig:eta_photo}. 
At $r - z > 0.6$, $\tage$ is underestimated, but this is solely a consequence
of the correlation between $r-z$ and true $\tage$. 
The simulated galaxies with $r - z > 0.6$ in our sample have overall older
stellar populations. 
In Figure~\ref{fig:etas_msfr}, we do not find a clear $M_*-{\rm SFR}$
dependence. 
However, $|\mu_{\Delta_{\tage}}|$ is larger and constraints are significantly
less precise for galaxies with older stellar populations below the star-forming sequence.  \\

\noindent \underline{\emph{Inferred $\tauism$}}:  
Lastly, we find that both the accuracy and precision of our $\tauism$ depend
significant only on the true $\tauism$ value. 
The inferred constraints increasingly underestimate $\tauism$ with lower
precision for greater $\tauism$.
The bias is due to discrepancies between the dust prescriptions of SPS model
and the mock observations. 
First, we use a dust prescription with a different attenuation curve in the SPS
model than in the forward model. 
This places a strict limit on how accurately we can derive $\tauism$.
We intentially introduce this discrepancy since we do not know the ``true''
attenuation curve of observed galaxies in practice. 
Another reason for the biased $\tauism$ constraints is that we only attenuate
the stellar emission in the disk component of the simulated galaxies and not
the bulge component (Section~\ref{sec:sed}).
The true $\tauism$ is the optical depth for the disk component while our
$\tauism$ constraints correspond to the optical depth of dust attenuation
for the entire galaxies, a quantity that will be lower than the true $\tauism$
depending on how much the bulge contributes to the SED. 
Hence, with these discrepancy we test whether the {\sc PROVABGS} SPS modeling
can marginalize over the effect of dust and derive robust constraints on the
other galaxy properties.

Despite the discrepancies, we find no significant SNR or color dependence on
the accuracy and precision of $\tauism$ constraints
(Figure~\ref{fig:eta_photo}). 
Furthermore, we find unbiased and precise $\tauism$ constraints for all galaxies
except star-forming galaxies above $M_* > 10^{11}M_\odot$ where we underestimate 
$\tauism$. 
Massive star-forming galaxies in this regime mainly have $\tauism > 1$.
In Figure~\ref{fig:etas}, we present a more apples-to-apples comparison of the
$\tauism$ constraints, where we present $\eta_\Delta$ for only galaxies without
bulge contributions (black dotted). 
For these galaxies, the bias in our $\tauism$ constraints is reduced and
$\mu_{\Delta_\theta}<0.5$ throughout the $\tauism$ range. 
Our constraints are still biased, however, due to the discrepant attenuation
curves. 
We emphasize that the primary goal of dust prescription in our SPS model is to
marginalize out the effect of dust. 
Based on the accuracy and precision of the constraints on other galaxy
properties, the {\sc PROVABGS} SPS model achieves this objective.  

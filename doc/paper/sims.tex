\section{Simulations}\label{sec:sims}
DESI, with its robotically-actuated and fibre-fed spectrographs, will collect 5000 spectra simultaneously. The spectra
cover the wavelength range 3600 to 9800 $\AA$, with a spectral resolution of $R
= \lambda/\Delta \lambda$ between 2000 and 5500.
During its five-year operation, starting in 2020, it will measure over 30
million spectra over $14,000~{\rm deg}^2$ of the sky (\todo{cite}). 
In addition, these DESI targets also have optical and infrared imaging data 
from the DESI Legacy Imaging Surveys~\citep[hereafter Legacy Surveys][]{dey2019}.
The Legacy Surveys are a combination of three public projects 
(Dark Energy Camera Legacy Survey, Beijing-Arizona Sky Survey, and Mayall 
$z$-band Legacy Survey) that jointly imaged the ${\sim}14,000~{\rm deg}^2$ DESI 
footprint in three optical bands ($g$, $r$, and $z$). Furthermore, DR8 of the
Legacy Survey also includes photometry in the WISE $W1$, $W2$, $W3$, and $W4$ 
infrared bands. \todo{The infrared photometry is
from all imaging through year 4 of NEOWISE-Reactivation force-photometered 
in the unWISE maps at the locations of Legacy Surveys optical sources (cite).}
Below we describe how we simulate realistic DESI-like galaxy spectra and photometry 
from state-of-the-art simulations.  


\begin{figure}
\begin{center}
\includegraphics[width=\textwidth]{figs/fm_photo.pdf}
\caption{{\em Left}: We forward model DESI $g$, $r$, and $z$ photometry (red)
    for our simulated galaxies (Section~\ref{sec:lgal}) by convolving their
    SEDs (dotted) with thei broadband filters (dashed) and then applying an 
    empirical noise model based on BGS objects in LS (Section~\ref{sec:photo}).
    {\em Right}: The $g-r$ and $r-z$ color distribution of the forward modeled
    \lgal~photometry is in good agreement with the color distribution of LS BGS
    objects (black contours).} \label{fig:photo}
\end{center}
\end{figure}

\subsection{L-Galaxies} \label{sec:lgal}
brief overview of {\sc L-Galaxies}~\citep[hereafter \lgal;][]{henriques2015}

\subsection{Spectral Energy Distributions} \label{sec:sed}
For each simulated galaxy, \lgal~provides the star formation histories (SFHs)
and chemical enrichment histories (ZH) for its bulge and disk components
separately, in approximately log-spaced lookback time bins. %15 lookback time bins approximately log-spaced between $t=0$ and $13.8$ Gyrs. 
We treat each lookback time bin, $i$, as a single stellar population (SSP) of
age $t_i$.
Then, we derive the luminosities of the bulge and disk components by summing up
the luminosities of all of their SSPs:
\begin{equation}
    L^{\rm comp.}(\lambda) = \sum \limits_i {\rm SFH}^{\rm comp.}_i \Delta t_i L_{\rm
    SSP}(\lambda; t_i, Z^{\rm comp.}_i). 
\end{equation}
${\rm SFH}^{\rm comp.}_i$ and $Z^{\rm comp.}_i$ are the star formation rate and
metallicity of the bulge or disk component in lookback time bin $i$. 
$\Delta t_i$ is the width of the bin. 
$L_{\rm SSP}$ corresponds to the luminosity of the SSP, which we calculate
using the Flexible Stellar Population Synthesis~\citep[\fsps][]{conroy2009,
conroy2010c} model.
% do we want to describe SPS models? 
For \fsps, we use the MIST ischrones~\citep{paxton2011, paxton2013, paxton2015,
choi2016, dotter2016}, the MILES spectral library~\citep{sanchez_blazquez2006},
and the \cite{chabrier2003} initial mass function (IMF). 

% applying velocity dispersion to bulge and disk components separately 
Next, we apply velocity dispersions to $L^{\rm comp.}(\lambda)$.
For the disk, we apply a fixed $50 km/s$ velocity dispersion; for the bulge, we
derive its velocity dispersion using the~\cite{zahid2016} empirical relation
that depends on the total bulge mass.
Then, we apply dust attenuation to stellar emission in the disk component
($L^{\rm disk}$) based on the cold gas content and orientation of the disk. 
We derive the attenuation curve using a mixed-screen model with the
\cite{mathis1983} dust extinction curve. 
Stellar emission from stars younger than $30{\rm Myr}$ are further attenuated
with a uniform dust screen and a wavelength dependent optical depth.
\todo{@rita further details and citations for the mixed-screen model}
No dust attenuation is applied to the bulge component.
\todo{@rita how come?}

Finally, we combine the attenuated disk component and the bulge component to
construct the total luminosity of the simulated galaxy and then convert this
rest-frame luminosity to observed-frame SED flux using its redshift, $z$.
\begin{equation}\label{eq:sed} 
    f_{\rm SED}(\lambda) = \frac{A(\lambda)L^{\rm disk}(\lambda) + L^{\rm bulge}(\lambda)}{4 \pi d_L(z)^2 (1+z)}.
\end{equation}
$A(\lambda)$ is the dust attenuation for the disk component described above 
and $d_L(z)$ is the luminosity distance.
In the left panel of Figure~\ref{fig:photo}, we present an example of the SED
flux constructed for an arbitrary \lgal galaxy (black).

\subsection{Forward Modeling DESI Photometry} \label{sec:photo} 
In this section, we describe how we construct realistic LS-like photometry
from the SEDs of simulated galaxies described in the last section.
First, we convolve the SEDs with the broadband filters of the Legacy Survey to
generate broadband photometric fluxes: 
\begin{equation} \label{eq:photo}
    f_X = \int f_{\rm SED}(\lambda) R_X(\lambda) {\rm d}\lambda
\end{equation}
where $f_{\rm SED}$ is the galaxy SED (Eq.~\ref{eq:sed}) and $R_X$ is the
transmission for filter $X$. 
We generate photometry for the $g$, $r$, and $z$ optical bands.
Next, we apply realistic measurement uncertainties to the derived photometry
using a simple empirical . 
We match each simulated galaxy to a BGS object from LS DR9 with the nearest $r$-band magnitude and
$g-r$ and $r-z$ color.
The photometric uncertainties ($\sigma_X$) and $r$-band fiber flux ($f_r^{\rm
fiber}$ of the BGS object are then assigned to the simulated galaxy. 
We sample a Gaussian distribution with standard deviation $\sigma_X$ and apply
it to construct realistic LS-like photometry: 
\begin{equation}
    \hat{f}_X = f_X + n_X  \quad {\rm where}~n_X \sim \mathcal{N}(0, \sigma_X).
\end{equation} 
Finally, we impose the target selection criteria of BGS~\citep[][Hahn~\etal~in
prep.]{ruiz-macias2021}.
In the left panel of Figure~\ref{fig:photo}, we overplot the forward
modeled photometry (red) ontop of the SED flux (black) for an arbitrary \lgal
galaxy. 
For reference, we also plot $R_X$ for the $g$, $r$, and $z$ bands of the Legacy
Survey in blue, orange, and green respectively. 
On the right panel, we compare the $g - r$ versus $r - z$ color distribution
for the forward modeled \lgal galaxies (red) to the color distribution of BGS
objects in LS (black contour). 
The forward modeled photometry show good agreement with LS BGS objects in
color space.

\begin{figure}
\begin{center}
\includegraphics[width=0.72\textwidth]{figs/fm_spec.pdf} \label{fig:spec}
\caption{
    We construct simulated DESI spectra (solid) for \lgal~simulated galaxies by
    applying a fiber aperture correction to the SED (dashed) and a realistic
    DESI noise model. 
    We apply a fiber aperture correction by scaling down the full SED (dotted)
    by the $r$-band fiber fraction derived from the Legacy Surveys imaging. 
    The noise model accounts for the DESI spectrograph response and an
    atmosphere model that accounts for the bright time observing conditions of
    BGS.
    Our forward model produces DESI-like spectra for all three arms of the DESI
    spectrographs: $b$, $r$, and $z$ (blue, orange, and green, respectively). 
    For more details, we refer readers to Section~\ref{sec:spec}.
    }
\end{center}
\end{figure}

\subsection{Forward Modeling DESI Spectra} \label{sec:spec}
In this section, we describe how we construct realistic DESI-like spectroscopy 
from the SEDs of simulated galaxies. 
Our forward model involves modeling the fiber aperture effect and applying a
noise model that accurately reproduces the bright time observations of BGS. 

DESI uses fiber-fed spectrographs with fibers that have angular radii of 1''. 
Only the light from a galaxy within this fiber aperture is collected by the 
instrument.
LS provides measurements of photometric fiber flux within a 1'' radius aperture
($f_X^{\rm fiber}$), which estimates the flux that passed through to the fibers.
When we assign photometric uncertainties to our simulated galaxies based on
$r$, $g-r$, and $r-z$ in Section~\ref{sec:photo}, we also assign $r$-band fiber
flux. 
We model the SED flux that passes through the fiber by scaling the SED flux by
the $r$ band fiber fraction, the ratio of the $r$-band fiber flux over the
total $r$ band flux: 
\begin{equation}
    f^{\rm spec}(\lambda) = \left(\frac{f_r^{\rm fiber}}{f_r}\right)f_{\rm SED}(\lambda).
\end{equation}
This fiber aperture correction assumes that there is no significant color
dependence. 
We also assume that there are no significant biases in the fiber flux
measurements in LS due to miscentering of objects. 
\todo{Do we want to say more about this assumption? Also cite Marta's paper
investigating aperture effect.}
In addition to using it for the aperture correction, we also use $f_r^{\rm
fiber}$ to derive ``measured'' $\hat{f}_r^{\rm fiber}$: 
\begin{equation}
    \hat{f}_r^{\rm fiber} = f_r^{\rm fiber} + n^{\rm fiber}_r \quad~{\rm
    where}~n^{\rm fiber}_r \sim \mathcal{N}\left(0, \frac{f_r^{\rm fiber}}{f_r}
    \sigma_r\right).
\end{equation}
We later use $\hat{f}_r^{\rm fiber}$ to set the prior on the nuisance parameter
of our SED modeling (Section~\ref{sec:methods}).

Next, we apply a noise model that simulates the DESI instrument response and
bright time observing conditions of BGS. 
We use the spectral simulation pipeline of \todo{cite survey simulations paper
and list details}
\footnote{\href{https://specsim.readthedocs.io/en/stable/guide.html}{https://specsim.readthedocs.io}}. 
More specifically, we use nominal dark time observing conditions with $180s$
exposure time. 
These conditions accurately reproduce the spectral noise and redshift success
rates of observed BGS exposures in DESI survey validation data (Hahn~\etal~in
prep.).
In Figure~\ref{fig:spec}, we present the forward modeled BGS spectrum of a
\lgal~galaxy (solid). 
For reference, we include the full SED (dotted) and fiber fraction scaled SED
(dashed) of the galaxy. 


%\todo{conclude by emphasizing the fact that our simulations cover the full expected
%observable space of DESI BGS and therefore if the pipeline works on our mocks,
%then it should work for every expected type of galaxies in the observations} 


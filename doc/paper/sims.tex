\section{Simulations}\label{sec:sims}
DESI, with its robotically-actuated and fibre-fed spectrographs, will collect 5000 spectra simultaneously. The spectra
cover the wavelength range 3600 to 9800 $\AA$, with a spectral resolution of $R
= \lambda/\Delta \lambda$ between 2000 and 5500.
During its five-year operation, starting in 2020, it will measure over 30
million spectra over $14,000~{\rm deg}^2$ of the sky (\todo{cite}). 
In addition, these DESI targets also have optical and infrared imaging data 
from the DESI Legacy Imaging Surveys~\citep[hereafter Legacy Surveys][]{dey2019}.
The Legacy Surveys are a combination of three public projects 
(Dark Energy Camera Legacy Survey, Beijing-Arizona Sky Survey, and Mayall 
$z$-band Legacy Survey) that jointly imaged the ${\sim}14,000~{\rm deg}^2$ DESI 
footprint in three optical bands ($g$, $r$, and $z$). Furthermore, DR8 of the
Legacy Survey also includes photometry in the WISE $W1$, $W2$, $W3$, and $W4$ 
infrared bands. \todo{The infrared photometry is
from all imaging through year 4 of NEOWISE-Reactivation force-photometered 
in the unWISE maps at the locations of Legacy Surveys optical sources (cite).}
Below we describe how we simulate realistic DESI-like galaxy spectra and photometry 
from state-of-the-art simulations.  


\begin{figure}
\begin{center}
\includegraphics[width=\textwidth]{figs/fm_photo.pdf}
\caption{{\em Left}: We forward model DESI photometry (red) for our simulated
    galaxies (Section~\ref{sec:lgal}) by convolving their SEDs with the
    broadband filters (dashed).  We construct the SEDs \ch{using the star
    formation and metallicity histories} (Section~\ref{sec:sed}).  We assign
    uncertainties to the forward modeled photometry by \ch{matching the
    colors to the Legacy Survey imaging for BGS} (Section~\ref{sec:photo}).
    {\em Right}: The $g-r$ and $r-z$ color distribution of the forward modeled
    {\sc LGal} photometry is consistent with the color distribution of BGS
    targets from Legacy Survey imaging (black contours).} \label{fig:photo}
\end{center}
\end{figure}

\subsection{LGal} \label{sec:lgal}
@rita

overview of Lgal. small volume semi-analytic and cosmo-hydro simulations

\todo{describe what galaxy properties (SFH, ZH, etc) are available} 

\subsection{Spectral Energy Distributions} \label{sec:sed}
\todo{describe how the SED is generated using the SFH and ZHs} 
In Fig.~\ref{fig:photo}, 

\subsection{Forward Modeled DESI Photometry} \label{sec:photo} 
We generate realistic Legacy Survey-like photometry for the galaxies in our
simulations using their noiseless source spectra or SEDs
(Section~\ref{sec:lgal}). We first generate {\em noiseless} photometry by
convolving the SEDs with the broadband filters of the Legacy Survey: 
\begin{equation}
     f_X = \int f(\lambda) R_X(\lambda) {\rm d}\lambda
\end{equation}
where $f(\lambda)$ is the galaxy SED and $R_X$ is the transmission for filter
$X$. We generate photometry for the $g$, $r$, and $z$ optical bands as well as
the $W1$ and $W2$ WISE infrared bands.

Next, to assign photometric uncertainties, $\sigma_X$, we match each simulated galaxy
to a Legacy Survey DR8 galaxy 
(\todo{specify how I'm defining galaxies here})
with the nearest $g$, $r$, and $z$ magnitudes as well as $g - r$ and $r - z$
colors. Afterwards, we use these uncertainties to assign ``measured''
photometry to the simulated galaxies by
sampling from a normal distribution centered at noiseless photometry with
variance set by the assigned uncertainties: $\hat{f}_X \sim \mathcal{N}(f_X,
\sigma_X)$. 

Finally, we impose the target selection criteria of BGS: $r < 20$. 
This leaves us with \todo{X} \lgal galaxies. 
In the left panel of Fig.~\ref{fig:photo}, we plot the forward
modeled photometry (red error bars) and the SED (black) for a \lgal galaxy. 
For reference, we also plot $R_X$ for the three optical bands of the Legacy
Survey. On the right panel, we compare the $g - r$ and $r - z$ color
distribution for the forward modeled \lgal galaxies (red) to the color
distribution of Legacy Survey galaxies (black contour). The two distributions
show good agreement.  

\subsection{Forward Modeled DESI Spectroscopy} \label{sec:spec}
fiber flux assigned when we assigned photometric uncertainty

\begin{itemize} 
    \item 'true' fiber aperture SED generated by scaling down the SED by the 
    assigned $r$-band fiber flux. 
    \item Meanwhile we add some noise to the fiber flux to have
    'measured' fiber flux. 
    \item Furthermore, we later jointly fit photometry and spectra we marginalize over $f_{\rm fiber}$.
    \item run true fiber aperture SED through the DESI noise model to produce DESI-like spectra. 
    \item describe the noise model in detail 
    \item \footnote{\href{https://specsim.readthedocs.io/en/stable/guide.html}{https://specsim.readthedocs.io}}
\end{itemize}


\begin{figure}
\begin{center}
\includegraphics[width=0.72\textwidth]{figs/fm_spec.pdf} 
\caption{We construct forward model DESI spectra (blue) for our simulated galaxies by first
applying a fiber aperture correction to the SED (black solid), then applying a DESI noise model. 
We apply a fiber aperture correction by scaling down the SED based on the $r$-band aperture
flux we assign from the Legacy Surveys imaging. The noise model includes a DESI instrument 
model, which accounts for the DESI spectrograph response, and an atmosphere model, which 
accounts for the bright time observing conditions. We construct simulated DESI spectra for 
8 different bright time observing conditions. For more details, we refer readers to Section~\ref{sec:spec}.
}
\label{fig:spec}
\end{center}
\end{figure}

A paragraph on how we combine these to get the spectrophotometry with figure 
{\bf figure} showing the spectra and photometry on top of each other 


\section{Simulations}\label{sec:sims}
DESI, with its robotically-actuated and fibre-fed spectrographs, will collect 5000 spectra simultaneously. The spectra
cover the wavelength range 3600 to 9800 $\AA$, with a spectral resolution of $R
= \lambda/\Delta \lambda$ between 2000 and 5500.
During its five-year operation, starting in 2020, it will measure over 30
million spectra over $14,000~{\rm deg}^2$ of the sky (\todo{cite}). 
In addition, these DESI targets also have optical and infrared imaging data 
from the DESI Legacy Imaging Surveys~\citep[hereafter Legacy Surveys][]{dey2019}.
The Legacy Surveys are a combination of three public projects 
(Dark Energy Camera Legacy Survey, Beijing-Arizona Sky Survey, and Mayall 
$z$-band Legacy Survey) that jointly imaged the ${\sim}14,000~{\rm deg}^2$ DESI 
footprint in three optical bands ($g$, $r$, and $z$). Furthermore, DR8 of the
Legacy Survey also includes photometry in the WISE $W1$, $W2$, $W3$, and $W4$ 
infrared bands. \todo{The infrared photometry is
from all imaging through year 4 of NEOWISE-Reactivation force-photometered 
in the unWISE maps at the locations of Legacy Surveys optical sources (cite).}
Below we describe how we simulate realistic DESI-like galaxy spectra and photometry 
from state-of-the-art simulations.  


\begin{figure}
\begin{center}
\includegraphics[width=\textwidth]{figs/fm_photo.pdf}
\caption{{\em Left}: We forward model DESI photometry (red) for our simulated
    galaxies (Section~\ref{sec:lgal}) by convolving their SEDs with the
    broadband filters (dashed).  We construct the SEDs \ch{using the star
    formation and metallicity histories} (Section~\ref{sec:sed}).  We assign
    uncertainties to the forward modeled photometry by \ch{matching the
    colors to the Legacy Survey imaging for BGS} (Section~\ref{sec:photo}).
    {\em Right}: The $g-r$ and $r-z$ color distribution of the forward modeled
    {\sc LGal} photometry is consistent with the color distribution of BGS
    targets from Legacy Survey imaging (black contours).} \label{fig:photo}
\end{center}
\end{figure}

\subsection{LGal} \label{sec:lgal}
@rita

overview of Lgal. small volume semi-analytic and cosmo-hydro simulations

\todo{describe what galaxy properties (SFH, ZH, etc) are available} 

\subsection{Spectral Energy Distributions} \label{sec:sed}
\todo{describe how the SED is generated using the SFH and ZHs} 
In Fig.~\ref{fig:photo}, 

\subsection{Forward Modeled DESI Photometry} \label{sec:photo} 
We generate realistic Legacy Survey-like photometry for the galaxies in our
simulations using their noiseless source spectra or SEDs
(Section~\ref{sec:lgal}). We first generate {\em noiseless} photometry by
convolving the SEDs with the broadband filters of the Legacy Survey: 
\begin{equation}
     f_X = \int f(\lambda) R_X(\lambda) {\rm d}\lambda
\end{equation}
where $f(\lambda)$ is the galaxy SED and $R_X$ is the transmission for filter
$X$. We generate photometry for the $g$, $r$, and $z$ optical bands as well as
the $W1$ and $W2$ WISE infrared bands.

Next, to assign photometric uncertainties, we match each simulated galaxy
to a Legacy Survey DR8 galaxy 
(\todo{specify how I'm defining galaxies here})
with the nearest $g$, $r$, and $z$ magnitudes as well as $g - r$ and $r - z$
colors. These uncertainties are then used to generate Gaussian noise and added
to $f_X$ to assign ``measured'' photometry to the simulated galaxy. 

Finally, we impose the target selection criteria of BGS: $r < 20$. 
This leaves us with \todo{X} \lgal galaxies. 
In the left panel of Fig.~\ref{fig:photo}, we plot the forward
modeled photometry (red error bars) and the SED (black) for a \lgal galaxy. 
For reference, we also plot $R_X$ for the three optical bands of the Legacy
Survey. On the right panel, we compare the $g - r$ and $r - z$ color
distribution for the forward modeled \lgal galaxies (red) to the color
distribution of Legacy Survey galaxies (black contour). The two distributions
show good agreement.  

\subsection{Forward Modeled DESI Spectroscopy} \label{sec:spec}
We're interested in exploiting the detailed information in DESI spectra in
addition to the photometry to improve constraints on the inferred physical parameters. 
We therefore also construct forward model DESI-like spectra for our simulated
galaxies. DESI fibers have 1'' radii, which means only the light within this
aperture is collected by the fiber. To simulate this fiber aperture, we apply
an aperture correction to the SEDs. 

Legacy Survey provides fluxes of objects within a 1'' radius aperture,
$f_X^{\rm fiber}$. This serves as an estimate the flux that passed through to
the fibers. Similar to how we assigned photometric uncertainties above, for
each simulated galaxy we identify a Legacy Survey galaxy with the nearest
$g$, $r$, and $z$ magnitudes and $g - r$ and $r - z$ colors. Then, we assign 
it the ratio of the $r$-band fiber flux to the total $r$-band flux 
($f_r^{\rm fiber}/f_r$) of this LS galaxy. We use the assigned
$f_r^{\rm fiber}/f_r$ as the aperture correction for the SED: 
\begin{equation}
    f^{\rm spec}(\lambda) = \left(\frac{f_r^{\rm fiber}}{f_r}\right)f(\lambda).
\end{equation}
This correction assumes there is no significant color dependence, we confirm
that this is the case for LS galaxies. We assume that there are no significant
biases in the LS $f_X^{\rm fiber}$ measurements from miscentering 
of objects. \todo{Do we want to say more about this assumption?}  

Besides the aperture correction, we also assign to each simulated galaxy a
``measured'' $\hat{f}_r^{\rm fiber}$ with the LS $f_r^{\rm fiber}$ plus a 
Gaussian noise based on the measured LS uncertainty for $f_r^{\rm fiber}$.
$\hat{f}_r^{\rm fiber}$ will later be used to set the prior for when we
marginalize over the aperture correction later in Section~\ref{sec:methods}.

Next, we generated realistic DESI-like spectra from the aperture corrected SEDs
using a forward model that simulates the DESI instrument response and realistic
observing conditions. \todo{details on desi instrument model}. \todo{details on
bgs observing conditions}. \todo{table of bgs observing conditions?}  
\footnote{\href{https://specsim.readthedocs.io/en/stable/guide.html}{https://specsim.readthedocs.io}}

In Figure~\ref{fig:spec}, we present 
\todo{conclude by emphasizing the fact that our simulations cover the full expected
observable space of DESI BGS and therefore if the pipeline works on our mocks,
then it should work for every expected type of galaxies in the observations} 

\begin{figure}
\begin{center}
\includegraphics[width=0.72\textwidth]{figs/fm_spec.pdf} 
\caption{We construct forward model DESI spectra (blue) for our simulated galaxies by first
applying a fiber aperture correction to the SED (black solid), then applying a DESI noise model. 
We apply a fiber aperture correction by scaling down the SED based on the $r$-band aperture
flux we assign from the Legacy Surveys imaging. The noise model includes a DESI instrument 
model, which accounts for the DESI spectrograph response, and an atmosphere model, which 
accounts for the bright time observing conditions. We construct simulated DESI spectra for 
8 different bright time observing conditions. For more details, we refer readers to Section~\ref{sec:spec}.
}
\label{fig:spec}
\end{center}
\end{figure}


\section{Simulations}\label{sec:sims}
In this Section, we describe how we construct mock observations from simulated
galaxies of the {\sc L-Galaxies} semi-analytic galaxy formation model (SAM).
We use a forward model that includes realistic noise, instrumental effects, and
observational systematics to produce DESI-like photometry and spectra. 
Later, we apply Bayesian SED modeling to these mock observations and
demonstrate that we can accurately infer the true galaxy propertries.

\begin{figure}
\begin{center}
\includegraphics[width=\textwidth]{figs/fm_photo.pdf}
\caption{{\em Left}: We forward model DESI optical $g$, $r$, and $z$ band
    photometry (red) for our simulated galaxies (Section~\ref{sec:lgal}) by
    convolving their SEDs (black dotted) with the broadband filters (dashed)
    and then applying an empirical noise model based on BGS objects in LS
    (Section~\ref{sec:photo}).
    {\em Right}: The $g-r$ and $r-z$ color distribution of the forward modeled
    \lgal~photometry is in good agreement with the color distribution of LS BGS
    objects (black contours). 
    \edits{We use 2,123 simulated galaxies from \lgal in total and plot a
    subsample of 425 in the color distribution for clarity.}
    } \label{fig:photo}
\end{center}
\end{figure}

\subsection{L-Galaxies} \label{sec:lgal}
{\sc L-Galaxies}~\citep[hereafter \lgal;][]{henriques2015} is a state-of-the-at
semi-analytic galaxy formation model run on subhalo merger trees from the
Millennium~\citep{springel2005a} and Millenium-II~\citep{boylan-kolchin2009}
$N$-body simulations. 
Millenium-I and II provide a dynamic range of $10^{7.0} < M_* < 10^{12}
M_\odot$ and adopts a \cite{planckcollaboration2014a} $\Lambda$CDM cosmology.
\lgal~includes prescriptions for gas infall and cooling, star formation, disc
and bulge formation, stellar and black hole feedback, and the environmental
effects of tidal and ram-pressure stripping.
Feedback from active galactic nuclei (AGN), which prevents hot gas from
cooling, is the major mechanism for quenching star formation in massive
galaxies.
\lgal~model parameters are calibrated against the observed stellar mass
functions and passive (quiescent) fractions at four different redshifts from 
$z = 3$ to 0.
We refer readers to \cite{henriques2015} for further detail on \lgal.

% A recent comparison to the cosmological hydrodynamical simulation
% IllustrisTNG,  \lgal was recently compared to stellar mass functions and the stellar masses of individual galaxies agree
%  to better than  ∼0.2  dex with IllustrisTNG (Mohammadreza2021)

\subsection{Spectral Energy Distributions} \label{sec:sed}
For each simulated galaxy, \lgal~provides the star formation histories (SFHs)
and chemical enrichment histories (ZH) for its bulge and disk components,
separately, in approximately log-spaced lookback time bins.  
We treat each lookback time bin, $i$, as a single stellar population (SSP) of
age $t_i$.
Then, we derive the luminosities of each component by summing up the
luminosities of their SSPs:
\begin{equation}
    L^{\rm comp.}(\lambda) = \sum \limits_i \left({\rm SFH}^{\rm comp.}_i
    \Delta t_i\right)~L_{\rm SSP}(\lambda\,;\, t_i, Z^{\rm comp.}_i). 
\end{equation}
${\rm SFH}^{\rm comp.}_i$ and $Z^{\rm comp.}_i$ are the star formation rate and
metallicity of the bulge or disk component in lookback time bin $i$. 
$\Delta t_i$ is the width of the bin. 
$L_{\rm SSP}$ corresponds to the luminosity of the SSP, which we calculate
using the Flexible Stellar Population Synthesis~\citep[\fsps;][]{conroy2009,
conroy2010c} model.
% do we want to describe SPS models? 
\edits{
    For \fsps, we use the MIST isochrones~\citep{paxton2011, paxton2013,
    paxton2015, choi2016, dotter2016} and the \cite{chabrier2003} initial mass
    function (IMF). 
    Also, we use the default spectral library in \fsps: the MILES spectral
    library~\citep{sanchez-blazquez2006} over the wavelength range
    $3800-7100\AA$ and the BaSeL
    library~\citep{lejeune1997, lejeune1998, westera2002} 
    outside of those limits.
}

% applying velocity dispersion to bulge and disk components separately 
Next, we apply velocity dispersions to $L^{\rm comp.}(\lambda)$.
For the disk, we apply a fixed $50\,\mathrm{km/s}$ velocity dispersion. 
For the bulge, we derive its velocity dispersion using the~\cite{zahid2016}
empirical relation that depends on the total bulge mass.
Afterwards, we apply dust attenuation to stellar emission in the disk component
($L^{\rm disk}$) based on the cold gas content and orientation of the disk. 
The attenuation curve is derived using a mixed-screen model with the
\cite{mathis1983} dust extinction curve. 
Stellar emission from stars younger than $30{\rm Myr}$ are further attenuated
with a uniform dust screen and a wavelength dependent optical depth.
No dust attenuation is applied to the bulge component.
We use the same dust attenuation that \cite{henriques2015} uses to construct
galaxy colors. 

Finally, we combine the attenuated disk component and the bulge component to
construct the total luminosity of the simulated galaxy and then convert this
rest-frame luminosity to observed-frame SED flux using its redshift, $z$.
\begin{equation}\label{eq:sed} 
    f_{\rm SED}(\lambda) = \frac{A(\lambda)L^{\rm disk}(\lambda) + L^{\rm bulge}(\lambda)}{4 \pi d_L(z)^2 (1+z)}.
\end{equation}
$A(\lambda)$ here is the dust attenuation for the disk component described
above and $d_L(z)$ is the luminosity distance.
In the left panel of Figure~\ref{fig:photo}, we present an example of the SED
flux constructed for an arbitrary \lgal~galaxy (black dotted).

\subsection{Forward Modeling DESI Photometry} \label{sec:photo} 
In this section, we describe how we construct realistic LS-like photometry
from the SEDs of simulated galaxies described in the last section.
First, we convolve the SEDs with the broadband filters of the LS to generate
broadband photometric fluxes: 
\begin{equation} \label{eq:photo}
    f_X = \int f_{\rm SED}(\lambda) R_X(\lambda) {\rm d}\lambda.
\end{equation}
$f_{\rm SED}$ is the galaxy SED (Eq.~\ref{eq:sed}) and $R_X$ is the
transmission curve for filter in the $X$ band. 
We generate photometry for the LS $g$, $r$, and $z$ optical bands.
Next, we apply realistic measurement uncertainties to the derived photometry by
roughly sampling the noise distribution of BGS targets from LS DR9. 
We do this by matching each simulated galaxy to a BGS target with the nearest 
$r$-band magnitude and $g-r$ and $r-z$ color.
The photometric uncertainties ($\sigma_X$) and $r$-band fiber flux ($f_r^{\rm
fiber}$) of the BGS object are then assigned to the simulated galaxy. 
We apply photometric noise by sampling a Gaussian distribution with standard
deviation $\sigma_X$: 
\begin{equation}
    \hat{f}_X = f_X + n_X  \quad {\rm where}~n_X \sim \mathcal{N}(0, \sigma_X).
\end{equation} 
Finally, we impose the target selection criteria of BGS~\citep[][Hahn~\etal~in
prep.]{ruiz-macias2021}.
In the left panel of Figure~\ref{fig:photo}, we overplot the forward
modeled photometry (red) on top of the SED flux (black) for an arbitrary
\lgal~galaxy. 
For reference, we also plot $R_X$ for the $g$, $r$, and $z$ bands of LS in
blue, orange, and green, respectively. 
On the right panel, we compare the $g - r$ versus $r - z$ color distribution
for the forward modeled \lgal~galaxies (red) to the color distribution of BGS
objects in LS (black contour). 
The errorbars represent the photometric uncertainties. 
The forward modeled photometry show good agreement with LS BGS targets in
optical color space.

\begin{figure}
\begin{center}
\includegraphics[width=0.72\textwidth]{figs/fm_spec.pdf}
\caption{
    We construct simulated DESI spectra (solid) for \lgal~simulated galaxies by
    applying a fiber aperture correction to the SED (dashed) and a realistic
    DESI noise model. 
    We apply a fiber aperture correction by scaling down the full SED (dotted)
    by the $r$-band fiber fraction derived from LS imaging. 
    The noise model accounts for the DESI spectrograph response and the bright
    time observing conditions of BGS (\ch{Hahn~\etal~in prep.,
    Schlafly~\etal~in prep.}).  
    We represent the spectra from the three arms ($b$, $r$, and $z$) of the
    DESI spectraphs in blue, orange, and green respectively. 
    Our forward model produces realistic DESI-like spectra that accurately
    reproduce the noise levels and characteristics of actual BGS spectra. 
    } \label{fig:spec}
\end{center}
\end{figure}

\subsection{Forward Modeling DESI Spectra} \label{sec:spec}
In the following, we describe how we construct realistic DESI-like spectroscopy 
from the SEDs of simulated galaxies. 
We begin forward modeling the fiber aperture effect. % and apply a noise model that accurately reproduces the bright time observations of BGS. 
DESI uses fiber-fed spectrographs with fibers that have angular radii of 1''. 
Hence, only the light from a galaxy within this fiber aperture is collected by
the instrument.
\edits{
    Among BGS targets in LS, 40\% have $r_e < 1''$ and 81\% have $r_e < 2''$ so
    the fiber aperture effect will significantly impact the majority of BGS
    galaxies.
    $r_e$ here is the half-light radius of the galaxy surface brightness model
    fit by {\sc Tractor}\footnote{http://thetractor.org/doc/}.
}
LS also provides measurements of photometric fiber flux within a 1'' radius aperture
($f_X^{\rm fiber}$), which estimates the flux that passes through to the fibers.
When we assigned photometric uncertainties to our simulated galaxies based on
$r$, $g-r$, and $r-z$ in Section~\ref{sec:photo}, we also assigned $r$-band
fiber flux. 
We model the SED flux that passes through the fiber (fiber loss) by scaling the 
SED flux by the $r$ band fiber fraction, the ratio of $f_r^{\rm fiber}$ over
the total $r$ band flux: 
\begin{equation}
    f^{\rm spec}(\lambda) = \left(\frac{f_r^{\rm fiber}}{f_r}\right)f_{\rm SED}(\lambda).
\end{equation}
This fiber aperture correction assumes that there is no significant color
dependence. 
It also assume that there are no significant biases in the fiber flux
measurements in LS due to miscentering of objects. 
We discuss the implications of these assumptions later in
Section~\ref{sec:discuss}. 
In addition to the aperture correction, we also use $f_r^{\rm fiber}$ to derive
``measured'' $\hat{f}_r^{\rm fiber}$: 
\begin{equation}
    \hat{f}_r^{\rm fiber} = f_r^{\rm fiber} + n^{\rm fiber}_r \quad~{\rm
    where}~n^{\rm fiber}_r \sim \mathcal{N}\left(0, \frac{f_r^{\rm fiber}}{f_r}
    \sigma_r\right).
\end{equation}
After all, when analyzing actual observations, we do not know the true fiber
fraction. 
We later use $\hat{f}_r^{\rm fiber}$ to set the prior on the nuisance parameter
of our SED modeling (Section~\ref{sec:methods}).

Next, we apply a noise model that simulates the DESI instrument response and
bright time observing conditions of BGS. 
We use the same noise model as the spectral 
simulations\footnote{\href{https://specsim.readthedocs.io/en/stable/guide.html}{https://specsim.readthedocs.io}} 
used for the BGS survey design and validation (Hahn~\etal~in prep.). 
We refer readers to \ch{Schlafly~\etal~(in prep.)} for details about the survey
simulations and \ch{Guy~\etal~(in prep.)} for details on the DESI spectroscopic
data reduction pipeline.
Specifically, we use nominal dark time observing conditions with $180s$
exposure time, which accurately reproduce the spectral noise and redshift
success rates of observed BGS spectra in DESI survey validation observations.
In Figure~\ref{fig:spec}, we present the forward modeled BGS spectrum of an
arbitrary \lgal~galaxy (solid). 
We mark the spectrum from each arm of the three DESI spectrographs separately 
(blue, orange, green).
For reference, we include the full SED (dotted) and fiber fraction scaled SED
(dashed) of the galaxy. 


%\todo{conclude by emphasizing the fact that our simulations cover the full expected
%observable space of DESI BGS and therefore if the pipeline works on our mocks,
%then it should work for every expected type of galaxies in the observations} 


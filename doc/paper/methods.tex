\section{Joint SED modeling of Photometry and Spectra} \label{sec:methods}
\subsection{Stellar Population Synthesis Modeling} \label{sec:sps} 
PROVABGS will provide inferred galaxy properties derived from joint SED
modeling of DESI photometry and spectra. 
For the SED modeling, we use a state-of-the-art stellar population synthesis
(SPS) model that uses a non-parametric SFH with a star-burst, a non-parametric
ZH that varies with time, and a flexible dust attenuation prescription. 

% describe SFH prescription
The form of the SFH is one of the most important factors in the accuracy of an
SPS model.
In general, the form of the SFH requires balancing between being flexible enough
to describe the wide range of SFHs in observations while not being too flexible
that it can describe any SFH at the expense of constraining power.  
If the model SFH is not flexible enough to describe actual SFHs of galaxies,
then unbiased galaxy properties cannot be inferred using the SPS model. 
For instance, most SPS models~(\emph{e.g.} CIGALE,~\citealt{serra2011};
BAGPIPES,~\citealt{carnall2017}) use parametric SFH such as the exponentially
declining $\tau$-model.
Such functional forms, however, produce biased estimates of galaxy properties
(\emph{e.g.} $M_*$ and SFR) when used to fit mock observations of simulated 
galaxies~\citep{simha2014, pacific2015, carnall2018}.
On the other hand, many non-parametric forms of the SFH are overly flexible
and allow unphysical SFHs~\citep{leja2019}, which unncessarily increases 
parameter degeneracies and discards constraining power. 

In our SPS model, we use a non-parametric SFH with two components: one based on
non-negative matrix factorization (NMF) basis functions and a starburst component.
For the first component, SFH is a linear combination of NMF SFH bases:
\begin{equation} \label{eq:nmf} 
    {\rm SFH}^{\rm NMF} (t, t_{\rm age}) = \sum\limits_{i=1}^{4} \beta_i
    \frac{s_i^{\rm SFH}(t)}{\int\limits_0^{t_{\rm age}} s_i^{\rm SFH}(t) \,
    {\rm d}t}. 
\end{equation} 
$\{s^{\rm SFH}_i\}$ are the NMF basis functions and $\{\beta_i\}$ are the
coefficients. 
The integral in the denominator normalizes the NMF basis functions to unity. 
We constrain $\sum_i \beta_i = 1$, so the total SFH of the component over the
age of the galaxy ($t_{\rm age})$ is normalized to unity.
$\{s^{\rm SFH}_i\}$ are from Tojeiro \etal~(in prep.) and derived from the
IllustrisTNG cosmological hydrodynamic simulation~\citep{nelson2018,
pillepich2018, springel2018}.
The SFHs of simulated galaxies IllustrisTNG are compiled, rebinnined, and smoothed
\todo{more details here}. 
Afterwards, we perform non-negative matrix
factorization~\citep{lee1999,cichocki2009, fevotte2011} on the smooth SFHs to
derive $\{s^{\rm SFH}_i\}$. 
We find that 4 components is sufficient to accurately reconstruct the SFHs
from IllustrisTNG. 
Assuming that the SFHs of IllustrisTNG galaxies resemble the SFHs of actual
observed galaxies, our NMF form provides a compact and flexible representation
of the SFHs. 

The NMF basis functions are derived from smooth SFHs, which means that it does
not include any stochasticity. 
However, observations and high resolution zoom-in hydrodyanmical simulations
both find significant stochasticity in galaxy SFHs~\citep{sparre2017,
caplar2019, hahn2019b, iyer2020}. 
To include this stochasticity in our SPS model, we include a starburst
component that consists of a SSP in the SFH. 
For the total SFH, we use
\begin{equation} \label{eq:sfh}
    {\rm SFH} (t, t_{\rm age}) = (1 - f_{\rm burst})~{\rm SFH}^{\rm NMF} (t,
    t_{\rm age}) + f_{\rm burst}~\delta_{\rm D}(t - t_{\rm burst}).
\end{equation}
$f_{\rm burst}$ is the fraction of total stellar mass formed during the
starburst; $t_{\rm burst}$ is the time at which the starburst occurs; 
$\delta_{\rm D}$ is the Dirac delta function.
In totatl we have 6 free parameters in our SFH: 4 NMF basis coefficients 
($\beta_i$), $f_{\rm burst}$, and $t_{\rm burst}$. 

% describe ZH 
Another key part of an SPS model is the ZH, or chemical enrichment history. 
Current SPS models mostly assume a ZH that does not vary over
time~\citep{carnall2017, leja2019}. 
Since galaxies in hydrodynamical simulations and observations do not have
constant metallicities throughout their history, this assumption can
significantly bias the inferred galaxy properties. 
Instead, for our ZH, we take a similar approach to the SFH and use NMF basis
functions:
\begin{equation}
    {\rm ZH}(t) = \sum\limits_{i=1}^2 \gamma_i s_i^{\rm ZH}(t).
\end{equation} 
$\{s_i^{\rm ZH}(t)\}$ are the ZH NMF basis functions and $\{\gamma_i\}$ are the
coefficients. 
$\{s_i^{\rm ZH}(t)\}$ are fit using the ZHs of simulated galaxies from IllustrisTNG. 
We use two NMF components, so our ZH prescription has 2 free parameters. 

\begin{figure}
\begin{center}
\includegraphics[width=0.8\textwidth]{figs/nmf_bases.pdf} 
    \label{fig:nmf}
    \caption{
        Non-negative matrix factorization basis functions for the SFH
        (left) and ZH (right). 
        These basis functions are derived from the SFHs and ZHs of simulated
        galaxies in the IllustrisTNG cosmological hydrodynamic simulations. 
    }
\end{center}
\end{figure}

Next, we use the SFH and ZH above to model the unattenuated rest-frame
luminosity as a linear combination of multiple SSPs, evaluated at
logarithmically-spaced lookback time bins.
We use a fixed log-binning with the bin egdes starting with $(0, 10^{6.05}{\rm
yr})$, $(10^{6.05}, 10^{6.15}{\rm yr})$, and continuing on with bins of width
0.1 dex.
The binning is truncated at the age of the model galaxy. 
For a $z=0$ galaxy, we use 43 $\tlb$ bins.
We use log-spaced $\tlb$ bins because it better reproduces galaxy luminosities
evaluated with much higher resolution $\tlb$ binning than linearly-spacing, for
the same number of bins. 
At every $\tlb$ bin $i$, we evaluate the luminosity of a SSP with ${\rm
ZH}(t_i)$, where $t_i$ is the center of $\tlb$ bin, and total stellar mass
calculated by resampling the SFH in Eq.~\ref{eq:sfh}. 
We use \fsps to evaluate the SSP luminosities and use the MIST ischrones, MILES
spectral library, and the \cite{chabrier2003} IMF (same as in
Section~\ref{sec:sed}).  
Since we use MIST isochrones, we impose a minimum and maximum limit to ${\rm
ZH}$ based on its coverage: $4.49\times10^{-5}$ and $4.49\times10^{-2}$,
respectively.
Our stellar metallicity range is significantly broader than previous studies,
\emph{e.g.} \cite{carnall2017, leja2017, tacchella2021}. 

Before we combine the SSP luminosities, we apply dust attenuation.
We use a two component \cite{charlot2000} dust attenuation model with birth
cloud (BC) and diffuse-dust (ISM) components. 
The BC component represents the extra dust attenuation of young stars that are
embedded in modecular clouds and HII regions. 
For SSPs younger than $t_i < 100{\rm Myr}$, we apply the
following BC dust attenuation: 
\begin{equation}
    L_i(\lambda) = L_i^{\rm unatten.}(\lambda) \exp\left[-\tau_{\rm BC} \left(
    \frac{\lambda}{5500\AA} \right)^{-0.7} \right].
\end{equation}
$\tau_{\rm BC}$ is the BC optical depth that determines the strength of the BC
attenuation. 
Afterwards, {\em all} SSPs are attenuated by the diffuse dust using the
\cite{kriek2013} attenuation curve parameterization: 
\begin{equation}
    L_i(\lambda) = L_i^{\rm unatten.}(\lambda) \exp\left[-\tau_{\rm ISM} \left(
    \frac{\lambda}{5500\AA} \right)^{n_{\rm dust}} \left(k_{\rm Cal}(\lambda) +
    D(\lambda) \right) \right].
\end{equation}
$\tau_{\rm ISM}$ is the diffuse dust optical depth.
$n_{\rm dust}$ is the \cite{calzetti2001} dust index, which determines the
slope of the attenuation curve. 
$k_{\rm Cal}(\lambda)$ is the \cite{calzetti2001} attenuation curve and
$D(\lambda)$ is the UV dust bump, parameterized using a Lorentzian-like Drude 
profile:
\begin{equation}
    D(\lambda) = \frac{E_b(\lambda~\Delta \lambda)^2}{(\lambda^2 -
    \lambda_0^2)^2 + (\lambda~\Delta \lambda)^2}
\end{equation}
where $\lambda_0 = 2175 \AA$, $\Delta \lambda = 350\AA$, and \todo{$E_b=?$} are the
central wavelength, full width at half maximum, and strength of the bump,
respectively. 
Once dust attenuation is applied to the SSPs, we sum them up to get the
rest-frame luminosity of the galaxy. 
In total, our SPS model has 12 free parameters: $M_*$, 4 SFH basis
coefficients, $f_{\rm burst}$, $t_{\rm burst}$, 2 ZH basis coefficients,
$\tau_{\rm BC}$, $\tau_{\rm ISM}$, and $n_{\rm dust}$. 

%description of our speculator SED model \citep{alsing2019}, which is based on FSPS. We use Chabrier IMF \ch{do we need to justify htis?}. 

In practice, evaluating each SSP using \fsps~requires \todo{X} seconds. 
For each model evaluation, we evaluate $\sim 43$ SSPs in each of the log-spaced
$\tlb$ bins; this takes \todo{X} seconds. 
Though this is not a prohibitive computational cost on its own, sampling a
high dimensional parameter space for inference requires $>100,000$ evaluations
--- \emph{i.e.} \todo{$>100$} CPU hours \emph{per galaxy}. 
For the $>10$ million BGS galaxies, this would require \todo{\emph{a billion}}
CPU hours. 
Instead, we use an emulator for the model luminosity, which uses a Principal
Component Analysis (PCA) neural network (NN) following the approach of
\cite{alsing2019}.

Our emulator consists of a NN and PCA basis functions. 
The NN provides a flexible and accurate mapping between the SPS model
parameters and PCA coefficients --- \emph{i.e.} the NN predicts PCA
coefficients for a given set of SPS parameters. 
Then the linear combination of the predicted coefficients and PCA basis
functions give us the emulated model luminosity. 
The PCA basis functions and NN are trained using $1,000,000$ pairs of SPS
parameters and model luminosity, $(\theta, L(\lambda;\theta))$. 
Throughout the wavelength range relevant for BGS, $3000 < \lambda < 9800\AA$,
we achieve $< 1\%$ accurate with the emulator. 
For details on the training, validation, and performance of our PCA NN
emulator, we refer readers to Kwon \etal~(in prep.). 

From the rest-frame luminosity, we obtain the observed-frame, redshifted, flux
in the same way as Eq.~\ref{eq:sed}.
In our case, redshift is not a free parameter since we will have spectroscopic
redshifts for every DESI BGS galaxy.
To model DESI photometry, we convolve the model flux with the LS broadband
filters as in Eq.~\ref{eq:photo}.
To model DESI spectra, we first apply Gaussian velocity dispersion. 
In this work, we keep velocity dispersion fixed at \todo{$0km/s$} but in
practice velocity disperion can be set as a free parameter. 
Then the broadened flux is resampled into the wavelength binning of the
observed DESI spectra, which has spectral resolution $R = 2000 - 5000$ over
$3600 < \lambda < 9800\AA$. 
Finally, the SPS model photometry and spectrum can be directly compared to
observations.


\begin{table} \label{tab:params}
\caption{Parameters of the PROVABGS SPS model used for joint SED modeling of
    DESI photometry and spectroscopy.} 
\begin{center}
    \begin{tabular}{ccc} \toprule
        name & description & prior \\[3pt]
        \hline 
        $\log M_*$                              & log galaxy stellar mass & uniform over [8, 13] \\
        $\beta_1, \beta_2, \beta_3, \beta_4$    & NMF basis coefficients for SFH & Dirichlet prior \\
        $f_{\rm burst}$ & fraction of total stellar mass formed in starburst event & uniform over [0, 1] \\
        $t_{\rm burst}$ & time of starburst event & uniform over [10Myr, 13.2Gyr] \\
        $\gamma_1, \gamma_2$ & NMF basis coefficients for ZH & uniform over [] \\
        $\tau_{\rm BC}$ & Birth cloud optical depth & uniform over [] \\
        $\tau_{\rm ISM}$ & diffuse-dust optical depth & uniform over [] \\
        $n_{\rm dust}$ & \cite{calzetti2001} dust index & unifrom over[]\\
        \hline            
\end{tabular} 
\end{center}
\end{table}


\subsection{Parameter Inference} \label{sec:infer} 
Using the SPS model above, we perform Bayesian parameter inference to derive
posterior probability distributions of the SPS parameters from photometry and
spectroscopy. 
From Bayes rule, we write down the posterior as
\begin{equation} \label{eq:bayes}
    p(\theta\given {\bf X}) \propto p(\theta)~p({\bf X} \given \theta)
\end{equation}
where ${\bf X}$ is the photometry or spectrum and $\theta$ is the set of SPS
parameters. 
$p({\bf X} \given \theta)$ is the likelihood, which we calculate separately for
the photometry
\begin{equation}
    \mathcal{L}^{\rm photo} \propto \exp\left[-\frac{1}{2} \left(\frac{X^{\rm photo} -
    m^{\rm photo}(\theta)}{\sigma^{\rm photo}}\right)\right]
\end{equation}
and for the spectrum
\begin{equation}
    \mathcal{L}^{\rm spec} \propto \exp\left[-\frac{1}{2} \left(\frac{X^{\rm spec} -
    m^{\rm spec}(\theta)}{\sigma^{\rm spec}} \right)^2\right].
\end{equation}
$m^{\rm photo}$ and $m^{\rm spec}$ represent SPS model for photometry and
spectroscopy. 
$\sigma^{\rm photo}$ and $\sigma^{\rm spec}$ respresent the uncertainties on
the measured photometry and spectrum. 
We consider the photometry indepedent from the spectrum so we combine the
likelihoods when jointly model the spectrophotometry: 
\begin{equation}
    \log \mathcal{L} \approx \log \mathcal{L}^{\rm photo} + \log
    \mathcal{L}^{\rm spec}.
\end{equation}
$p(\theta)$ is the prior on the SPS parameters. 
For most of our parameters, we use uninformative uniform priors with
conservatively chosen ranges that are listed in Table~\ref{tab:param}. 
However, for the priors of $\{\beta_1, \beta_2, \beta_3, \beta_4 \}$, the NMF coefficients
for the SFH, we use a Dirichlet distribution.  
With a Dirichlet distribution, $\beta_i$ are within $0 < \beta_i < 1$ and
satisfy the constraint $\sum_i \beta_i = 1$. 
This maintains the normalization of the SFH in Eq.~\ref{eq:nmf}. 

Now that we can evaluate the posterior using the likelihood and prior, we
derive the posterior distributions using Markov Chain Monte Carlo (MCMC)
sampling. 
We use the \cite{karamanis2020} ensemble slice sampling MCMC algorithm with the
{\sc zeus} Python package. 
\todo{highlight advantages of ensemble slice MCMC}

When we sample the posterior, we do not directly sample our 12 dimensional
SPS parameter space. 
This is because we impose a Dirichlet prior on the SFH NMF coefficients. 
Dirichlet distributions are difficult to directly and efficiently sample so
instead we use the \cite{betancourt2012} sampling method, which allows us to
transforms an $N$ dimensional Dirichlet distribution into an easier to sample
$N-1$ dimensional space.
Hence, we sample the posterior in the transformed 11 dimensional space. 

%go into detail about the Dirichlet priors and the slight complication with sampling that. 
James's work in convergence here. 
talk about how in
principle speculator can easily be used with HMC because you can get
derivatives with backpropagation. 
provide detailed profiling of SED fitting and projections for full 10million galaxy BGS sample. 

\begin{figure}
\begin{center}
    \includegraphics[width=0.95\textwidth]{figs/mcmc_posterior_demo.pdf}
    \label{fig:posterior}
    \caption{
        \emph{Top}: 
        Posterior probability distribution of our 12 SPS model parameters
        derived from joint SED modeling of the mock DESI photometry and
        spectra.
        The contours mark the 68 and 95\% percentiles of the distribution. 
        \emph{Bottom}: 
        Comparison of our SPS model evaluated at the best-fit parameters
        (orange) with the mock observations (black). 
        On the left panel, we compare the $g$, $r$, $z$ band magnitudes; on the
        right, we compare the spectroscopy.  
    }
\end{center}
\end{figure}

In Figure~\ref{fig:posterior} we present the posterior as well as the best-fit photometry and
spectroscopy using speculator + emcee. 

%%%%%%%%%%%%%%%%%%%%%%%%%%%%%%%%%%%%%%%%%%%
%% parameter table
%%%%%%%%%%%%%%%%%%%%%%%%%%%%%%%%%%%%%%%%%%%
%\begin{table}
%\caption{Table of the different parameterizations for the spectro-photometric fitting. The changes in each setup with respect to default configuration, CIGALE D, are bolted. } 
%\begin{center} 
%\begin{tabular}{ccccc} \toprule
%name & SFH & dust & IMF \\[3pt]
%\hline 
%iFSPS C & rita's basis & & \\
%Firefly  & rita's basis & & \\
%VESPA    & & & \\
%CIGALE A & \textbf{delayed SFH} & Charlot $\&$ Fall 2000, Draine et al. 2014 & Salpter \\
%CIGALE B & delayed SFH + additional burst & \textbf{Calzetti et al. 2000, Dale et al. 2014} & Salpter \\
%CIGALE C & delayed SFH + additional burst & Charlot $\&$ Fall 2000, Draine et al. 2014 & \textbf{Chabrier} \\
%CIGALE D & delayed SFH + additional burst & Charlot $\&$ Fall 2000, Draine et al. 2014 & Salpeter \\[3pt]
%\hline            
%\end{tabular} \label{tab:setups}
%\end{center}
%\end{table}
%%%%%%%%%%%%%%%%%%%%%%%%%%%%%%%%%%%%%%%%%%%

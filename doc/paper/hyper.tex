\section{Population Inference} \label{sec:hyper}
% restatement/high level explanation of what we do
We quantify the accuracy and precision of the inferred galaxy properties from
our SED modeling using population hyperparameters 
$\eta_{\Delta} = \{\mu_{\Delta_\theta}, \sigma_{\Delta_\theta}\}$ 
(Section~\ref{sec:results}). 
These hyperparameters describe the distribution of the difference between the
inferred and true parameters, $\Delta_{\theta}$, assuming that the distribution
has a Gaussian functional form (Eq.~\ref{eq:eta_gauss}). 
The $\eta_\Delta$ values we present in this work are MAP estimates of
$p(\eta_\Delta | \{ X_i \})$, the probability distribution of $\eta_\Delta$
given some galaxy population observations. 
They are inferred using population inference as described in the main text and
Eqs~\ref{eq:popinf} - \ref{eq:popinf2}.
Our approach for quantifying the accuracy and precision has a number of key
advantages over other methods. 
For instance, a naive way to quantify the accuracy and precision would be to
estimate the median and standard deviations of individual posteriors then
averaging them. 
This assumes that each individual posterior is close to a Gaussian. 
As we later demonstrate, this is an incorrect assumption that reduces the
posterior distribution to point estimates. 
Another approach would be to stack the posteriors by summing up all of the
individual posteriors. 
Neither of these approaches mathematically estimate the distribution we are
actually interested in estimating: $p(\eta_\Delta | \{ X_i \})$. 
Moreover, both approaches are biased. 
\cite{malz2020} recently demonstrated this in the context of combining
photometric redshift posteriors. 

\begin{figure}
\begin{center}
\includegraphics[width=0.45\textwidth]{figs/etas_demo.pdf}
    \caption{
    The $\log M_*$ distribution described by the accuracy and precision
    hyperparameters for galaxies with $10.6 < \log M_* < 10.8$:
    $\mathcal{N}(10.7 + \mu_{\Delta_\theta}, \sigma_{\Delta_\theta})$ 
    (black dashed).
    The hyperparameters are MAP estimates of $p( \mu_{\Delta_\theta}, 
    \sigma_{\Delta_\theta} | \{ X_i \})$ derived from population inference
    (Eq.~\ref{eq:popinf}-\ref{eq:popinf2}). 
    We include individual $\log M_*$ posteriors of several galaxies with $M_*
    \sim 10^{10.7} M_\odot$ for comparison.   
    The individual posteriors have a wide variety of shapes, which can bias 
    naive estimates of their accuracy and precision. 
    The comparison illustrates that 
    $\mathcal{N}(\mu_{\Delta_\theta}, \sigma_{\Delta_\theta})$ provides a
    robust estimate of the overall accuracy and precision of the inferred
    posteriors. 
    }\label{fig:eta_demo}
\end{center}
\end{figure}
We illustrate the population inference approach in Figure~\ref{fig:eta_demo} 
where we present the distribution of $\log M_*$ described by the accuracy and
precision hyperparameters derived for galaxies with $10.6 < \log M_* < 10.8$: 
$\mathcal{N}(\mu_{\Delta_\theta} + 10.7, \sigma_{\Delta_\theta})$ (black
dashed). 
For comparison, we plot posteriors of $\log M_*$ for several individual
galaxies with $\log M_* \sim 10.7$.  
There is significant variation in the individual posteriors and many of them
are not well described by a Gaussian distribution. 
This variation is an expected consequence of noise in the observables and
MCMC sampling.  
We note that estimating the accuracy and precision by stacking the posteriors,
for instance, significantly underestimates the precision.  
Meanwhile, the accuracy and precision hyperparameters capture the overall
accuracy and precision of the individual posteriors. 

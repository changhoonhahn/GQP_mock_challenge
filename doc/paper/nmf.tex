\section{Non-negative Matrix Factorization Bases} \label{sec:nmf}
The basis vectors for the star-formation and metallicity histories are computed
using non-negative matrix factorisation (NMF) on a set of star formation and
metallicity histories in the Illustris 
simulation~\citep{vogelsberger2014, genel2014, nelson2015}.
Unlike PCA, NMF lends itself well to this task as it gives positive vectors,
which can each be straightforwardly interpreted physically as representing the
SFH of a composite stellar population. 
In the case of the ZHs, the advantage of NMF over PCA is less clear, but we
maintain the NMF scheme for simplicity. 

The SFHs and ZHs are computed from all stellar particles bound to subhalos
that host a galaxy with $M_* > 10^9 M_\odot$ at $z=0$, giving a sample of just
over 29,000 Illustris galaxies. 
For the SFHs, we take the distribution of stellar ages in 400 bins,
logarithmically distributed between 8.6 Myrs and 13.65 Gyrs, and compute the
stellar mass formed in each bin. For the ZHs, we take the mass-weighted
metallicity in each of the bins. 
Next, the vectors for the SFHs and ZHs are normalized independently ---
\emph{i.e.} we do not keep information of which ZH corresponds to each SFH.
Therefore we do not impose the mass-metallicity relation of the simulation onto
our basis vectors (see \citealt{thorne2021} for a parameterization that links
SFH with ZH throught he mass-metallicity relation). 
We take each set of simulated SFHs and ZHs as a reasonable representation of
possible SFHs and ZHs in the Universe. 
Prior to decomposition, each individual vector is smoothed on a scale of 400
Myr, which removes any information on smaller timescales. 
We decompose the set of SFHs into 4 independent components, and the set of ZHs
into 2 independent components. 
The resulting components are shown in the main text (Figure~\ref{fig:nmf}). 

\begin{figure}
\begin{center}
\includegraphics[width=0.85\textwidth]{figs/NFMrec_gal2000_A.pdf}
\includegraphics[width=0.85\textwidth]{figs/NFMrec_gal2000_B.pdf}
    \caption{
    The original and NMF-reconstructed SFHs (top left) and ZHs (top right) of a
    galaxy in the Illustris simulation. 
    The original SFH is shown after smoothing on a scale of 400 Myr. 
    We mark the contributions of each of the NMF components in the faded
    colored lines. 
    The middle and bottom panels compare the spectra obtained from integrating
    the original and reconstructed SFH and ZH. 
    In this case, the NMF basis offers a good reconstruction of the SFH and ZH,
    which results in a small residuals in the corresponding spectra.
    }\label{fig:nmf0}
\end{center}
\end{figure}

\begin{figure}
\begin{center}
\includegraphics[width=0.85\textwidth]{figs/NFMrec_gal9009_A.pdf}
\includegraphics[width=0.85\textwidth]{figs/NFMrec_gal9009_B.pdf}
    \caption{
    Same as Figure~\ref{fig:nmf0} but for another Illustris galaxy. 
    In this case, the NMF basis fails to reproduce a burst of star formation at
    recent times, leading directly to an underestimation of the luminosity,
    especially towards the bluer wavelengths.
    }\label{fig:nmf1}
\end{center}
\end{figure}

\begin{figure}
\begin{center}
\includegraphics[width=0.85\textwidth]{figs/ensemble_plots_4p2.jpg}
    \caption{
    Comparison of original versus NMF reconstructed total stellar mass formed,
    mass-weighted age, mass-weighted metallicity, and mass formed in the last
    200 Myr (from top to bottom). 
    On the left column we present the distribution of each quantity, and on the
    right column we show direct comparisons in scatter plots. 
    The orange dashed lines shows the one-to-one line.
    Total stellar mass is well recovered, as expected given its lack of
    sensitivity to smaller bursts. 
    Mass-weighted ages are poorly recovered at young and old ages, as a direct
    consequence of the lack of resolution of our basis.
    The mass-weighted metallicity is well recovered on the mean, though with a
    large scatter. 
    The mass formed in young stars is again affected by the lack of resolution
    of our basis. 
    In our SPS model, we include a stochastic burst component to account for
    this limitation (Section~\ref{sec:sps}). 
    }\label{fig:nmf2}
\end{center}
\end{figure}

Figures~\ref{fig:nmf0} and~\ref{fig:nmf1} show two examples of the NMF direct
reconstruction on two galaxies. 
The two galaxies are chosen as examples of a `fair' and a `poor' reconstruction.
In all cases the reconstructions can be improved by increasing the number of
components, and doing so effectively improves our ability to model shorter
timescale features in the SFH and ZHs.
In this work, we instead include a stochastic burst component in the SFH
(Section~\ref{sec:sps}). 

In Figure~\ref{fig:nmf2}, we present how NMF reconstruction projects onto
certain derived properties: total stellar mass formed, mass-weighted age,
mass-weighted metallicity and mass in young stars (mass formed in the last 200
Myr).
Besides the total stellar mass, the other derived properties are impacted by
the lack of short timescale features. 
Our stochastic burst component directly addresses this limitiation. 
Therefore, the NMF basis can be seen as a reasonable and minimal set to recover
the broad shape of the star-formation and metallicity histories, which is
complemented in our SPS model by the stochastic burst component.

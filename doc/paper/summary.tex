\section{Summary}
Over the next five years, DESI will measure spectra for ${>}30$ million
galaxies, including ${>}10$ million galaxies in BGS during bright time.
Each DESI galaxy will also have optical photometry from the Legacy Survey. 
BGS, which will extend out to $z\sim0.6$, will provide a $r < 19.5$
magnitude-limited sample of ${\sim}10$ galaxies spanning a wide range of galaxy
properties with high completeness and ${>}95\%$ redshift efficiency. 
It will also include a sample of fainter galaxies down to $r < 20.175$ selected
based on a fiber magnitude and color. 
This upcoming dataset offers a unique opportunity to leverage its statistical
power for galaxy evolution and maximize its scientific impact. 
For instance, having measured galaxy properties for such a statistically
powerful sample of galaxies would enable us to measure population statistics
and empirical relations of galxaies with unprecedented precision. 
It would also enable more complete and precise comparisons between observations
and galaxy formation models, which will shed light into the physical processes
of galaxy evolution.
To exploit this opportunity, we will construct the PRObabilistic Value-Added
Bright Galaxy Survey (PROVABGS), where we will apply state-of-the-art Bayesian
SED modeling to jointly analyze DESI spectroscopy and LS photometry. 
PROVBGS will provide full posterior distributions of galaxy properties, such as
stellar mass ($M_*$), star formation rate (SFR), stellar metallicity 
($Z_{\rm MW}$), and stellar age ($\tage$), for ${>}10$ million BGS galaxies.

In this work, we present and validate the SED model, Bayesian inference
framework, and other methods that will be used to construct PROVABGS.
We use \todo{2239} galaxies in the {\sc L-Galaxies} semi-analytic model to
construct realistic synthetic DESI spectra and photometry.  
We build SEDs using stellar population synthesis based on the star formation
and chemical enrichment histories of the simulated galaxies.
Then we simulate the SEDs using the forward modeling pipeline used in the BGS
survey design.  
Afterwards, we apply the PROVABGS SED modeling on the mock DESI observations to
derive posteriors on $M_*$, $\avgsfr$, $\zmw$, and $\tage$. 
From the posteriors and the population inference we conduct to quantify the
accuracy and precision, we find: 
\begin{itemize}
    \item Overall we derive posteriors on galaxy properites that are in good
        agreement with the true properties of the simulated galaxies. 
        Furthermore, with posteriors rather than point estimates we accurately
        estimate the uncertainties on the galaxy properties. 
        We infer posteriors with the following levels of precision: 
        $\sigma_{M_*}\sim0.1$dex, $\sigma_{\log\avgsfr}\sim0.1$dex, 
        $\sigma_{\log\zmw}\sim0.15$dex, and $\sigma_{\tage}\sim0.5$ Gyr. 
        Our results also demonstrate that we successfully marginalize over the
        effect of dust and other nuisance parameters. 
    \item Like any SED model, the PROVABGS SED model also imposes significantly
        non-uniform priors on galaxy properties. 
        We find that these priors impose a lower bound on $\avgsfr$ of 
        $\avgsfr > 10^{-1}M_\odot/{\rm yr}$. 
        It also biases $\zmw$ by ${\sim}0.3$dex for observations with low
        spectral signal-to-noise and imposes an upper bound of $\tage < 8$ Gyr
        on $\tage$. 
        We characterize the priors in detail so that constraints on galaxy
        properties can be interpreted in future use of PROVABGS. 
    \item We compare the posteriors derived from DESI spectrophotometry to
        those derived from photometry alone. 
        Including DESI spectra substantially improves the constraints on galaxy
        properties. 
        In fact, jointly analyzing spectra is {\em essential} for mitigating
        the impact of the SED model priors. 
        For example, with photometry alone, the priors impose a more
        restrictive $\avgsfr > 1 M_\odot/{\rm yr}$ lower bound. 
\end{itemize}

We demonstrate with our mock challenge that we will derive accurate and precise
constraints on galaxy properties in PROVABGS. 
Beyond $M_*$, $\avgsfr$, $\zmw$, and $\tage$, which we focus on in this work, 
PROVABGS will also constrain star formation and metallicity histories. 
With galaxy properties of over ${>}10$ BGS galaxies, current galaxy studies
will be able to use the PROVABGS catalog to exploit the statistical power of
BGS for the most precise measurements of various galaxy relations. 
Furthermore, since the BGS samples span a wide range of galaxies, PROVABGS will
also enable galaxy studies to investigate less explored regimes, such as the
low mass galaxy populations. 
PROVABGS will be a fully probabilistic catalog with posteriors for all the
properties that accurately capture their uncertainties.
With these posteriors, we can also conduct more rigorous statistical analyses
using new techinques such as population inference and Bayesian hierarchical
inference.
We use one such method, population inference, to estimate the overall accuracy
and precision of our galaxy property constraints. 
These methods will not only improve the accuracy of our analyses but will allow
us to fully exploit the statistical power of DESI observations. 

Despite the overall success of the PROVABGS methodologies, as demonstrated in
this mock challenge, there are some limitations. 
For instance, we only consider a simple model for the effect of the DESI fiber
aperture and flux calibration. 
We reserve a more detailed investigation for Ramos~\etal~(in prep.). 
We also do not consider varying the isochrones, stellar library, or IMF. 
Instead, we will release multiple versions of PROVABGS with different sets of
assumptions. 
Lastly, we find that the most significant limitation to deriving accurate
galaxy properties comes from the prior imposed by the SED model. 
We will address this limitation and present a method to impose uniform priors
on galaxy properties in the upcoming Hahn~\etal~(in prep.). 

DESI has started its main 5 year operation. 
Already, as part of survey validation, DESI has collected \todo{X} spectra of
BGS galaxies that will be released in the Survey Validation Data Assembly
(SVDA). 
The SVDA release will also be accompanied by papers describing the data
reduction pipeline, redshift fitting algorithm, fiber assignment, survey
operation and simulations, visual inspection, and target selection for the
various tracers. 
Finally, using BGS observations in the SVDA, we will construct and release the
PROVABGS-SV catalog and present the stellar mass function measured from it in
the subsequent paper. 

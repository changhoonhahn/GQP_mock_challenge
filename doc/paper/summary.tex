\section{Summary}
Over the next five years, DESI will measure spectra for ${>}30$ million
galaxies, including ${>}10$ million galaxies in BGS during bright time.
In addition, each DESI galaxy will also have optical photometry from the Legacy
Survey. 
This upcoming dataset offers a unique opportunity to leverage its statistical
power for galaxy evolution and maximize its scientific impact. 
For instance, having measured galaxy properties for such a statistically
powerful sample of galaxies would enable us to measure population statistics
and empirical relations of galxaies with unprecedented precision. 
It would also enable more complete and precise comparisons between observations
and galaxy formation models, which will shed light into the physical processes
of galaxy evolution.
To exploit this opportunity, we will construct the PRObabilistic Value-Added
Bright Galaxy Survey (PROVABGS), where we will apply state-of-the-art Bayesian
SED modeling to jointly analyze DESI spectroscopy and LS photometry. 
PROVBGS will provide full posterior distributions of galaxy properties, such as
stellar mass ($M_*$), star formation rate (SFR), stellar metallicity 
($Z_{\rm MW}$), and stellar age ($\tage$), for ${>}10$ million BGS galaxies.


In this work, we present and validate the SED model, Bayesian inference
framework, and other methods that will be used to construct PROVABGS.
We use \todo{2239} galaxies in the {\sc L-Galaxies} semi-analytic model to
construct realistic synthetic DESI spectra and photometry.  
We build SEDs using stellar population synthesis based on the star formation
and chemical enrichment histories of the simulated galaxies.
Then we simulate the SEDs using the forward modeling pipeline used in the BGS
survey design.  
Afterwards, we apply the PROVABGS SED modeling on the mock DESI observations to
derive posteriors on $M_*$, $\avgsfr$, $\zmw$, and $\tage$. 
From the posteriors and the population inference we conduct to quantify the
accuracy and precision, we find: 
\begin{itemize}
    \item Overall we derive posteriors on galaxy properites that are in good
        agreement with the true properties of the simulated galaxies. 
        Furthermore, with posteriors rather than point estimates we accurately
        estimate the uncertainties on the galaxy properties. 
        We infer posteriors with the following levels of precision: 
        $\sigma_{M_*}\sim0.1$dex, $\sigma_{\log\avgsfr}\sim0.1$dex, 
        $\sigma_{\log\zmw}\sim0.15$dex, and $\sigma_{\tage}\sim0.5$ Gyr. 
        Our results also demonstrate that we successfully marginalize over the
        effect of dust and other nuisance parameters. 
    \item Like any SED model, the PROVABGS SED model also imposes significantly
        non-uniform priors on galaxy properties. 
        We find that these priors impose a lower bound on $\avgsfr$ of 
        $\avgsfr > 10^{-1}M_\odot/{\rm yr}$. 
        It also biases $\zmw$ by ${\sim}0.3$dex for observations with low
        spectral signal-to-noise and imposes an upper bound of $\tage < 8$ Gyr
        on $\tage$. 
        We characterize the priors in detail so that constraints on galaxy
        properties can be interpreted in future use of PROVABGS. 
    \item We compare the posteriors derived from DESI spectrophotometry to
        those derived from photometry alone. 
        We find that including spectra substantially improves the constraints
        on galaxy properties. 
        In fact, including spectra is {\em essential} for mitigating the impact
        of the SED model priors. 
        For example, with photometry alone, the priors impose a more
        restrictive $\avgsfr > 1 M_\odot/{\rm yr}$ lower bound. 
\end{itemize}
\todo{impact of PROVABGS} 

paragraph summarizing the limitations but in a hopeful way. 

Paragraph listing all the follow up papers: 
Prior correction paper,
Marta's aperture correction paper,
announce the PROVABGS-SV paper 

\section{Summary}
Over the next five years, DESI will measure spectra for ${>}30$ million
galaxies, each with optical photometry from the Legacy Surveys. 
BGS, which will extend out to $z\sim0.6$, will provide a $r < 19.5$
magnitude-limited sample of ${\sim}10$ million galaxies spanning a wide range
of galaxy properties with high completeness.
It will also include a sample of ${\sim}5$ million fainter galaxies down to $r
< 20.175$ selected based on a fiber magnitude and color. 
This upcoming dataset offers a unique opportunity to leverage its statistical
power for galaxy evolution and maximize its scientific impact. 
Accurate galaxy properties for such a galaxy sample, for instance,  would
enable us to measure population statistics and empirical relations of galaxies
with unprecedented precision. 
It would also enable more complete and precise comparisons between observations
and galaxy formation models, which will shed light into the physical processes
of galaxy evolution.
To exploit this opportunity, we will construct the PRObabilistic Value-Added
Bright Galaxy Survey (PROVABGS) catalog, where we will apply state-of-the-art
Bayesian SED modeling to jointly analyze DESI photometry and spectroscopy. 
PROVABGS will provide full posterior distributions of galaxy properties, such as
stellar mass ($M_*$), star formation rate (SFR), stellar metallicity 
($\zmw$), and stellar age ($\tage$), for all ${>}10$ million BGS galaxies.

In this work, we present and validate the SED model, Bayesian inference
framework, and other methodology that will be used to construct
PROVABGS\footnote{publicly available at
\href{https://github.com/changhoonhahn/provabgs/}{https://github.com/changhoonhahn/provabgs/}}.
We use 2,123 galaxies in the {\sc L-Galaxies} semi-analytic model to construct
realistic synthetic DESI spectra and photometry.  
We build SEDs using SPS based on the star formation and chemical enrichment
histories of the simulated galaxies.
Then, we simulate the SEDs using the forward modeling pipeline used in the BGS
survey design.  
Afterwards, we apply the PROVABGS SED modeling on the mock DESI observations to
derive posteriors on $M_*$, $\avgsfr$, $\zmw$, and $\tage$. 
From the posteriors and the population inference we conduct to quantify
accuracy and precision, we find: 
\begin{itemize}
    \item Overall, we derive posteriors of galaxy properties that are in good
        agreement with the true properties of the simulated galaxies. 
        Furthermore, with posteriors rather than point estimates we accurately
        estimate the uncertainties on the galaxy properties. 
        We infer posteriors with the following levels of precision: 
        $\sigma_{\log M_*}\sim0.1$ dex, $\sigma_{\log\avgsfr}\sim0.1$ dex, 
        $\sigma_{\log\zmw}\sim0.15$ dex, and $\sigma_{\tage}\sim0.5$ Gyr. 
        Our results also demonstrate that we successfully marginalize over the
        effect of dust and other nuisance parameters. 
    \item Like any SED model, the PROVABGS SED model imposes significantly
        non-uniform priors on galaxy properties. 
        We find that these priors impose a lower bound on $\avgsfr$ of 
        $\avgsfr > 10^{-1}M_\odot/{\rm yr}$. 
        It also biases $\zmw$ by ${\sim}0.3$ dex for observations with low
        spectral signal-to-noise and imposes an upper bound of $\tage < 8$ Gyr. 
        We characterize the priors in detail so that constraints on galaxy
        properties can be interpreted in future studies that use PROVABGS.
    \item We compare the posteriors derived from DESI spectrophotometry to
        those derived from photometry alone. 
        Including DESI spectra substantially improves the constraints on galaxy
        properties. 
        Moreover, jointly analyzing spectra is {\em essential} for mitigating
        the impact of the SED model priors. 
        For example, with photometry alone, the priors impose a more
        restrictive $\avgsfr > 1 M_\odot/{\rm yr}$ lower bound and bias $\zmw$
        ${\sim}0.5$ dex.
\end{itemize}

We demonstrate with our mock challenge that we will derive accurate and precise
constraints on specific galaxy properties in PROVABGS. 
Beyond $M_*$, $\avgsfr$, $\zmw$, and $\tage$, which we focus on in this work, 
PROVABGS will also constrain star formation and metallicity histories. 
With galaxy properties of over ${>}10$ million BGS galaxies, current galaxy
studies will be able to use the PROVABGS catalog to exploit the statistical
power of BGS for the most precise measurements of various galaxy relations.
Since the BGS samples span a wide range of galaxies, PROVABGS will
also enable galaxy studies to investigate less explored regimes, such as dwarf
galaxy populations. 

Furthermore, PROVABGS will be a fully probabilistic catalog.
With posteriors for all the galaxy properties, we can conduct more rigorous
statistical analyses using new techinques such as population inference and
hierarchical Bayesian modeling.
We demonstrate one such approach in this work by using population inference to
estimate the overall accuracy and precision of our galaxy property constraints. 
These methods will not only improve the accuracy of our analyses but they will
also allow us to fully exploit the statistical power of DESI observations. 

Despite the overall success of the PROVABGS methodologies that we demonstrate,
there are some limitations. 
For instance, we only consider a simple model for the effect of the DESI fiber
aperture and flux calibration. 
A more detailed investigation will be presented in Ramos~\etal~(in prep.). 
We also do not consider varying the isochrones, stellar library, or IMF. 
Instead, we will release multiple versions of PROVABGS with different sets of
assumptions. 
Lastly, we find that the most significant limitation to deriving accurate
galaxy properties comes from the prior imposed by the SED model. 
We will address this limitation and present a method to impose uniform priors
on galaxy properties in Hahn~(in prep.). 

DESI has started its main 5 year operation. 
Already, as part of survey validation, DESI has collected over 400,000 spectra
of BGS galaxies that will be released in the Survey Validation Data Assembly
(SVDA). 
The SVDA release will also be accompanied by papers describing the data
reduction pipeline, redshift fitting algorithm, fiber assignment, survey
operation and simulations, visual inspection, and target selection for the
various tracers. 
Finally, using BGS observations in the SVDA, we will construct and release the
PROVABGS-SV catalog and present the probabilistic stellar mass function
measured from it in the subsequent paper. 


The entire PROVABGS SED modeling pipeline, including the neural emulators and
Bayesian inference framework, is publicly available at:
\url{https://github.com/changhoonhahn/provabgs/}.
All of the software and scripts used in our analysis are publicly available at:
\url{https://github.com/changhoonhahn/gqp_mc}.
The accompanying data used in this work, including the mock DESI observations
and posteriors derived from PROVABGS, is available at:
\url{https://doi.org/10.5281/zenodo.5910635}.

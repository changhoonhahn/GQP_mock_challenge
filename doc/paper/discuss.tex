\section{Discussion} \label{sec:discuss}
We demonstrate in this work that we can derive accurate and precise posteriors
on specific galaxy properties using the {\sc PROVABGS} SED modeling. 
We infer full Bayesian posteriors on galaxy properties, unlike many previous
works that derive ``best-fit'' point estimates from maximizing the likelihood.
Posteriors have crucial advantages over maximum likelihood estimates. 
For instance, maximum likelihood estimates do \emph{not} provide uncertainties
on the galaxy properties. 
These uncertainties are significant, especially for properties such as $\zmw$
(Figure~\ref{fig:prop_inf}), and ignoring them overestimates the statistical
precision of the derived galaxy proprties. 
Although, some maximum likelihood methods provide estimates on the
uncertainties~\citep[\emph{e.g.}][]{moustakas2013, boquien2019}, 
they require strong assumptions on the shape of the likelihood. 
Hence they are inaccurate when there are significant parameter degeneracies or
when the posterior distribution has multiple modes.
This is the case for our SED modeling, where we find a multimodal posterior
distribution with \emph{e.g.} degeneracies between $f_{\rm burst}$ and
$\{\beta_1, \beta_2, \beta_3, \beta_4\}$ or between 
$\{\gamma_1, \gamma_2\}$ and $\{\beta_1, \beta_2, \beta_3, \beta_4\}$
(Figure~\ref{fig:posterior}). 
Maximum likelihood methods cannot capture these
complexities~\citep[\emph{e.g.}][]{boquien2019} and, therefore, lead
to overall less accurate constraints on galaxy properties.  
Furthermore, as we discuss further in this section, the posteriors on galaxy
properties enable us to correct for the priors imposed on galaxy properties by
our SPS model.
They also open the door for rigorous population inference and hierarchical
Bayesian approaches. 

\todo{not sure if this belongs here} 
In addition to the demonstrating the accuracy and precision of the inferred
galaxy properties, our results also highlight the advantages of including
spectra in SED modeling. 
Galaxy spectra provide precise spectroscopic redshifts, which is essential for
deriving meaningful constraints on galaxy properties. 
However, beyond providing redshift, we find that spectra provide substantial
statistical power for constraining galaxy galaxy properties. 
For instance, including spectra in the SED modeling significantly reduces the
bias of our $\zmw$ and $\tage$ constraints (Figure~\ref{fig:etas}). 
It also reduces the lower bound on the inferred $\avgsfr$. 
In fact, without spectra, we are dominated by priors on $\avgsfr$ and cannot
robustly infer galaxy properties of quiescent galaxies with $\log\avgsfr > 0$
dex.
We emphasize that all of these benefits come from spectra with the SNR of BGS,
which is observing during bright time. 
With ${>}10$ million galaxy spectra of BGS, {\sc PROVABGS} will go beyond other
galaxy surveys provide a unprecedented statistically powerful value-added
galaxy catalog. 

\begin{figure}
\begin{center}
\includegraphics[width=0.7\textwidth]{figs/sfh_demo.pdf}
    \caption{
        With the {\sc PROVABGS} SPS model, we can infer posteriors on the full
        star formation and metallicity histories from observations. 
        We present the inferred SFH and ZH for arbitrarily chosen star-forming
        galaxy (blue) and quiescent galaxy (orange) from our \lgal~sample.
        The shaded region represent the 64 and 95\% confidence intervals of the
        SFH and ZH posteriors. 
        For comparison, we include the true SFH and ZH (dashed). 
        The inferred SFH and ZH show good agreement with the true values;
        however, similar to the inferred $\avgsfr$ and $\zmw$, SFH and ZH are
        significantly impacted by priors imposed by the SPS model. 
    } \label{fig:sfh_demo}
\end{center}
\end{figure}


%\todo{Beyond the galaxy properties we discuss in Section~\ref{sec:results}, we can also derive SFH and ZH}
In this work, we primarily focus on the following physical properties of
galaxies: $\log M_*$, $\log\avgsfr$, $\log\zmw$, $\tage$, and $\tauism$. 
The {\sc PROVABGS} SPS model, however, can constrain galaxy properties beyond
these properties. 
The SPS model employs nonparametric SFH and ZH prescriptions based on NMF bases
and the model parameters include coefficients for these bases. 
Posteriors on the SPS model parameters, thus, can be used to derive constraints
on the SFH and ZH. 
In Figure~\ref{fig:sfh_demo}, we present the inferred SFH and ZH of two
simulated galaxies from our \lgal~sample: a star-forming galaxy (blue) and a
quiescent galaxy (orange). 
We mark the 68 and 95\% confidence intervals in the shaded regions. 
For comparison, we include the true SFH and ZH from \lgal~(dashed).  
The inferred SFH and ZH is able to generally recover the true histories. 
We emphasize that current SPS models assume constant ZHs that does not vary
over time~\citep{carnall2017, leja2019}. 
Hence inferring ZH over time is one of the key advantages of our {\sc PROVABGS}
SPS model. 
Similar to the inferred $\avgsfr$ and $\zmw$, the SFH and ZH constraints are
also impacted by the priors imposed by our SPS model. 
We charaterize the impact of the prior in detail in
Appendix~\ref{sec:model_priors}.

The most significant limitation of the {\sc PROVABGS} SED modeling in
accurately inferring the true galaxy properties is the prior on galaxy
properties imposed by the model. 
The effect of such priors is  a major limitation for any SED modeling
methods~\citep{carnall2017, leja2019}. 
It is a consequence of the fact that galaxy properties are \emph{not}
parameters of the SPS model.
For instance, $\avgsfr$ and $\zmw$ are derived by integrating the SFH and ZH
(Eq.~\ref{eq:prop_eqs}), which are parameterized by $\beta_1, \beta_2, \beta_3,
\beta_4$ and $\gamma_1, \gamma_2$. 
Hence, uniform priors on $\beta$s and $\gamma$s (Section~\ref{sec:infer} and
Table~\ref{tab:params}) do not translate into uniform priors on $\avgsfr$ and
$\zmw$.
Other galaxy properties (\emph{e.g.}~$\tage$, SFH, and ZH) likewise have
non-uniform, and undesireable, priors. 

One way to address this issue  is to choose an SED model that does not impose
extreme priors on galaxy properties and characterize the priors in detail so
that final posteriors can be appropriately  interpreted. 
For our {\sc PROVABGS} model, we explicitly chose our SFH prescription so that
the prior on $\log\avgssfr$ extends the range $-12$ to $-9$ dex.
Furthermore, in Appendix~\ref{sec:model_priors} and
Figures~\ref{fig:model_priors} and~\ref{fig:sfh_prior}, we fully characterize
the prior on $\avgssfr$, $\zmw$, $\tage$, SFH, and ZH.
However, we can go beyond solely minimizing the impact of the priors. 
With an estimate of the prior distribution, we can impose maximum-entropy
priors in a specified distribution~\citep{handley2019}. 
From an estimate of the prior distribution on the galaxy properties, we can
derive a new prior on the SPS model parameters that would impose uniform priors
on the galaxy properties. 
In Hahn in preparation, I will demonstrate that with this new prior, we can
infer posteriors on derived galaxy properties without being significantly
impacted by their prior. 
With this prior correction, we will be able to infer even more accurate
posteriors on the physical properties of galaxies with our {\sc PROVABGS} SED
modeling.

In this work, we use forward modeled mock observations to demonstrate that we
can infer accurate and precise posteriors on certain galaxy properties.
The mock observations are constructed from \lgal~and include photometry and
spectra. 
In generating the spectra, we model the fiber aperture effect --- \emph{i.e.}
spectra only include light from a galaxy collected within the DESI fiber
aperture --- by scaling the SED flux (Section~\ref{sec:spec}).
We account for this fiber aperture effect in the {\sc PROVABGS} SED model using 
a normalization factor, $f_{\rm fiber}$ (Section~\ref{sec:sps}). 
Hence, our mock observations and SED modeling have a consistent treatment of
the fiber aperture effect. 
In observations, however, fiber aperture effects can be wavelength dependent
(\todo{relevant citation}). 
For spectra affected by strongly wavelength dependent fiber aperture effects,
an overall $f_{\rm fiber}$ factor would not be sufficient.
In order to estimate the wavelength dependence of the fiber aperture
effect, we compare the ratio of the predicted fiber flux over flux from LS
photometry in  $g$, $r$, $z$ bands. 
Fiber flux is the predicted photometric flux within a fiber of diameter 1.5''
so the ratio estimates the fiber aperture effect for a specific photometric band. 
Based on the comparison with LS photometry, we find little wavelength
dependence in the fiber aperture effect. \todo{something more concrete after I
redo the comparison}

In addition to the fiber aperture effect, flux calibration performed on DESI
spectra can also induce wavelength dependent residuals. 
DESI spectra is measured using three-arm spectrographs that split the spectra
into three $b$, $r$, and $z$ channels with overlapping wavelength ranges: 
$3600 - 5930$, $5660 - 7720$, and $7470 - 9800 \AA$.  
After sky subtraction, flux calibration is performed on each channel of the
spectra by matching physical stellar models to spectra of spectrohotometric
standard stars observed in the same exposure. 
\todo{some concrete number about how well flux calibration works}. 
Despite the overall excellent performance of DESI flux calibration, since the
calibration is performed for each channel seaparately, inaccuracies can imprint
a wavelength dependent residual. 
In Ramos \etal~(in preparation), we examine the fiber aperture effect and
wavelength dependent imprints on DESI spectra using BGS spectra from the DESI
Survey Validataion data and MaNGA observations. 
\todo{some statement about how we will devise a correction or marginalization
accordingly}. 
We note that the {\sc PROVABGS} SED modeling pipline already includes flux
calibration models beyond a single $f_{\rm fiber}$. 
It can also easily be extended to include more sophisticated models such as a
Chebyschev polynomial flux calibration model used in other SED
models~\citep[\emph{e.g.}][]{carnall2017, tacchella2021}. 

\begin{itemize}
    \item caveats: theoretical assumptions --- isochrones and stellar
        libraries, summary of the appendix
\end{itemize}

\todo{PROVABGS discussion  paragraph} 
\begin{itemize}
\item despite the caveats, this work demonstrates that the galaxy
    properties inferred will be robust and awesome. Paragraph that mentions
    the extension of all the previous science applicatoins 
\item paragraph that discusses the new science applications with the
    posteriors. 
\end{itemize}

\section{Discussion} \label{sec:discuss}
We demonstrate in this work that we can derive accurate and precise posteriors
on galaxy properties using our {\sc PROVABGS} SED modeling. 

\todo{posterior over MAP}

\todo{reiterate and discuss advantages of spectra+photometry that emphasizes
why BGS will be awesome}


\begin{figure}
\begin{center}
\includegraphics[width=0.8\textwidth]{figs/sfh_demo.pdf}
    \caption{
        With the {\sc PROVABGS} SPS model, we can infer posteriors on the full
        star formation and metallicity histories from observations. 
        We present the inferred SFH and ZH for arbitrarily chosen star-forming
        galaxy (blue) and quiescent galaxy (orange) from our \lgal~sample.
        The shaded region represent the 64 and 95\% confidence intervals of the
        SFH and ZH posteriors. 
        For comparison, we include the true SFH and ZH (dashed). 
        The inferred SFH and ZH show good agreement with the true values;
        however, similar to the inferred $\avgsfr$ and $\zmw$, SFH and ZH are
        significantly impacted by priors imposed by the SPS model. 
    } \label{fig:sfh_demo}
\end{center}
\end{figure}


%\todo{Beyond the galaxy properties we discuss in Section~\ref{sec:results}, we can also derive SFH and ZH}
In this work, we primarily focus on the following physical properties of
galaxies: $\log M_*$, $\log\avgsfr$, $\log\zmw$, $\tage$, and $\tauism$. 
The {\sc PROVABGS} SPS model, however, can constrain galaxy properties beyond
these properties. 
The SPS model employs nonparametric SFH and ZH prescriptions based on NMF bases
and the model parameters include coefficients for these bases. 
Posteriors on the SPS model parameters, thus, can be used to derive constraints
on the SFH and ZH. 
In Figure~\ref{fig:sfh_demo}, we present the inferred SFH and ZH of two
simulated galaxies from our \lgal~sample: a star-forming galaxy (blue) and a
quiescent galaxy (orange). 
We mark the 68 and 95\% confidence intervals in the shaded regions. 
For comparison, we include the true SFH and ZH from \lgal~(dashed).  
The inferred SFH and ZH is able to generally recover the true histories. 
We emphasize that current SPS models assume constant ZHs that does not vary
over time~\citep{carnall2017, leja2019}. 
Hence inferring ZH over time is one of the key advantages of our {\sc PROVABGS}
SPS model. 
Similar to the inferred $\avgsfr$ and $\zmw$, the SFH and ZH constraints are
also impacted by the priors imposed by our SPS model. 
We charaterize the  

Of course, like the galaxy properties we present in this work, the SFH and ZH
constraints are also subject to the priors imposed by our SPS model. 


\todo{model priors and preview of how we can correct for it}

\todo{caveat paragraph} 
\begin{itemize}
    \item caveats: flux calibration --- what was implemented versus what will
    need to be implemented in observations
    \item caveats: theoretical assumptions --- isochrones and stellar
        libraries, summary of the appendix
\end{itemize}

\todo{PROVABGS discussion  paragraph} 
\begin{itemize}
\item despite the caveats, this work demonstrates that the galaxy
    properties inferred will be robust and awesome. Paragraph that mentions
    the extension of all the previous science applicatoins 
\item paragraph that discusses the new science applications with the
    posteriors. 
\end{itemize}

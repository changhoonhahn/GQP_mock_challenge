\section{Discussion} \label{sec:discuss}
The most significant limitation of the {\sc PROVABGS} SED modeling in
inferring the true galaxy properties is the prior on galaxy properties imposed
by the model. 
The effect of such priors is  a major limitation for any SED modeling
methods~\citep[\emph{e.g.}][]{carnall2017, leja2019}. 
It is a consequence of the fact that galaxy properties are \emph{not}
parameters of the SPS model.
For instance, $\avgsfr$, $\zmw$, and $\tage$ are derived by integrating the SFH
and ZH (Eq.~\ref{eq:prop_eqs}), which are parameterized by $\beta_1, \beta_2,
\beta_3, \beta_4$ and $\gamma_1, \gamma_2$. 
Uniform priors on $\beta$s and $\gamma$s (Section~\ref{sec:infer} and
Table~\ref{tab:params}) do not translate into uniform priors on $\avgsfr$,
$\zmw$ and $\tage$.
Other galaxy properties (\emph{e.g.}~SFH, and ZH) likewise have
non-uniform, and undesireable, priors. 

One way to address this issue is to choose an SED model parameterization that
does not impose extreme priors on galaxy properties and to characterize the
priors in detail so that final posteriors can be appropriately interpreted. 
For the {\sc PROVABGS} model, we explicitly chose our SFH prescription so that
the prior on $\log\avgssfr$ extends the range $-12$ to $-9$ dex.
Furthermore, we fully characterize the prior on $\avgssfr$, $\zmw$, $\tage$,
SFH, and ZH in Appendix~\ref{sec:model_priors} (Figures~\ref{fig:model_prior}
and~\ref{fig:sfh_prior}). 
This way, we understand exactly how the model prior impacts the derived
posteriors as we discuss in detail in Section~\ref{sec:results}. 
Beyond mitigating the effect of the priors, we can impose uniform prior (or any
other desired prior distribution) on the derived galaxy properties by adjusting
the priors on the SED model parameters. 
\cite{handley2019} recently demonstrated that maximum-entropy priors can be
used for this purpose to impose uniform priors on the inferred sum of neutrino
masses in cosmological analyses. 
In an upcoming paper, Hahn (in prep.), I will demonstrate that maximum-entropy
priors can also be used in Bayesian SED modeling to correct for the impact of
priors on infer posteriors on derived galaxy properties. 

%With this prior correction, we will be able to infer even more accurate posteriors on the physical properties of galaxies with our {\sc PROVABGS} SED modeling.

%However, we can go beyond minimizing the impact of the priors and use maximum-entropy.  With an estimate of the prior distribution, we can impose maximum-entropy priors in a specified distribution~\citep{handley2019}.  From an estimate of the prior distribution on the galaxy properties, we can derive a new prior on the SPS model parameters that would impose uniform priors on the galaxy properties. 

In this work, we use forward modeled mock observations to demonstrate that we
can infer accurate and precise posteriors on certain galaxy properties.
The mock observations are constructed from \lgal~and include photometry and
spectra. 
In generating the spectra, we model the fiber aperture effect --- \emph{i.e.}
spectra only include light from a galaxy collected within its fiber diameter 
--- by scaling the SED flux (Section~\ref{sec:spec}).
In our SED modeling, we account for this fiber aperture effect using a
normalization factor, $f_{\rm fiber}$ (Section~\ref{sec:sps}). 
Hence, our mock observations and SED modeling have a consistent treatment of
the fiber aperture effect. 
In observations, however, aperture effects can be wavelength
dependent~\citep{gerssen2012, richards2016}, and if the dependence is strong,
an overall $f_{\rm fiber}$ factor would not be sufficient.
In order to examine the wavelength dependence, we compare the ratio of the
fiber aperture flux over total flux in $g$, $r$, and $z$ bands of BGS targets
from LS.
We find find no significant difference in the flux ratios of the different
bands, which suggests that the fiber aperture effect in DESI does not have a
strong wavelength dependence. 

Flux calibration performed on DESI spectra can also induce wavelength dependent
residuals. 
DESI spectra are measured using three-arm spectrographs that split the spectra
into three $b$, $r$, and $z$ channels with overlapping wavelength ranges: 
$3600 - 5930$, $5660 - 7720$, and $7470 - 9800 \AA$.  
After flat fielding and sky subtraction, flux calibration is performed on each
channel of the spectra by matching physical stellar models to spectra of
spectrohotometric standard stars observed in the same exposure
(\ch{Guy~\etal~in prep.}). 
Since the calibration is performed for each channel seaparately, imperfections
can imprint a wavelength dependent residual. 
In a subsequent paper, Ramos \etal~(in prep.), we examine the fiber aperture
effect and wavelength dependent imprints on DESI spectra using BGS
spectra from the DESI Survey Validation data and observations from the Mapping
Nearby Galaxies at APO (MaNGA) survey. 
Using galaxy properties derived using the {\sc PROVABGS} pipeline for spectra
from integrated field unit MaNGA observations, we will present aperture
corrections that can be applied on derived BGS galaxy properties. 
We also note that the {\sc PROVABGS} SED modeling pipeline already includes flux
calibration models beyond a single $f_{\rm fiber}$ and can easily be extended
to include more sophisticated models~\citep[\emph{e.g.} Chebyschev
polynomial;][]{carnall2017, tacchella2021}. 

%\todo{paragraph on how we handle theoretical assumptions --- isochrones and stellar libraries, summary of the appendix}
In both our {\sc PROVABGS} SED model and mock observations, we use the MIST
isochrones, 
\edits{
    combined MILES+BaSeL spectral library,
}
and the \cite{chabrier2003} IMF.
With the same set of choices, our analysis does not consider how different
choices for stellar evolution or IMF can affect the inferred galaxy properties. 
Yet, it is well-established that there are major uncertainties in each of these
choices~\citep{conroy2009, conroy2013}.
For instance, recent observational works suggest that there may be significant
variations in IMF~\citep[\emph{e.g.}][]{treu2010, vandokkum2010, rosani2018,
sonnenfeld2019}. 
Different SPS model choices can significantly impact the derived galaxy
propeties~\citep[\emph{e.g.}][]{ge2019}.
We reserve a detailed examination of this effect for future work. 
In the meantime, for the {\sc PROVABGS} catalog we will release multiple
catalogs each with different sets of choices for isochrone, spectral library,
and IMF.

We demonstrate with the mock challenge that we can derive accurate and precise
constraints on specific galaxy properties using the {\sc PROVABGS} SED modeling.
The {\sc PROVABGS} catalog will have a number of key advantages over other
value-added galaxy catalogs. 
First, {\sc PROVABGS} will provide full Bayesian posteriors on galaxy
properties instead of ``best-fit'' point estimates from maximizing the
likelihood. 
Posterior distributions are essential for accurately estimating uncertainties
on galaxy properties.  
As we find earlier, these uncertainties are significant, especially for
properties such as $\zmw$ (Figure~\ref{fig:prop_inf}). 
Ignoring the uncertainties dramatically overestimates the statistical precision
of the derived galaxy properties and can significantly bias any galaxy study.
%We also note that the maximum-entropy method, mentioned earlier, to correct for the effect of priors on derived galaxy properties requires full posterior distributions.

Furthermore, the {\sc PROVABGS} posteriors will be derived from MCMC sampling
rather than grid-based methods often used in the
past~\citep[\emph{e.g.}][]{dacunha2008, moustakas2013, boquien2019}.
As a result, they can accurately estimate posterior distributions with
significant parameter degeneracies or multiple modes (peaks). 
For instance, in the posterior of Figure~\ref{fig:posterior} we find
degeneracies between $f_{\rm burst}$ and $\{\beta_1, \beta_2, \beta_3,
\beta_4\}$ and between $\{\gamma_1, \gamma_2\}$ and $\{\beta_1, \beta_2,
\beta_3, \beta_4\}$. 
The posterior is also multi-modal. 
Accurate estimates of the full posterior distribution are especially important,
as they enable the maximum-entropy method, mentioned earlier, 
to correct for the significant impact of priors on derived galaxy properties.
Grid-based methods also scale exponentially with the number of SPS parameters
so they quickly become infeasible as the dimensionality of SPS models increase. 
Meanwhile, MCMC scales approximately linearly with the number of parameters. 

\begin{figure}
\begin{center}
\includegraphics[width=0.7\textwidth]{figs/sfh_demo.pdf}
    \caption{
        With the {\sc PROVABGS} SPS model, we can infer posteriors on the full
        star formation and metallicity histories. 
        We present the inferred SFH and ZH for an arbitrarily chosen
        star-forming (blue) and quiescent galaxy (orange).
        The shaded region represent the 64 and 95\% confidence intervals of the
        SFH and ZH posteriors. 
        For comparison, we include the true SFH and ZH (dashed). 
        The inferred SFH and ZH show good agreement with the true values;
        however, similar to the inferred $\avgsfr$ and $\zmw$, the SFH and ZH
        are significantly impacted by priors imposed by the SPS model. 
    } \label{fig:sfh_demo}
\end{center}
\end{figure}
%\todo{Beyond the galaxy properties we discuss in Section~\ref{sec:results}, we can also derive SFH and ZH}
In this work, we primarily focus on the following physical properties of
galaxies: $\log M_*$, $\log\avgsfr$, $\log\zmw$, $\tage$, and $\tauism$. 
The {\sc PROVABGS} SPS model, however, can constrain galaxy properties beyond
these properties. 
The SPS model employs nonparametric SFH and ZH prescriptions based on NMF bases
and the model parameters include coefficients for these bases. 
Posteriors on the SPS model parameters can, thus, be used to derive constraints
on the SFH and ZH. 
In Figure~\ref{fig:sfh_demo}, we present the inferred SFH and ZH of two
simulated galaxies from our \lgal~sample: a star-forming (blue) and a quiescent
galaxy (orange). 
We mark the 68 and 95\% confidence intervals in the shaded regions. 
For comparison, we include the true SFH and ZH from \lgal~(dashed).  
The inferred SFH and ZH is able to generally recover the true histories. 
We emphasize that current SPS models typically assume constant ZHs that does
not vary over time~\citep{carnall2017, leja2019}. 
Hence inferring ZH over time is a key advantage of the {\sc PROVABGS} SPS
model. 
Similar to the inferred $\avgsfr$ and $\zmw$, the SFH and ZH constraints are
also impacted by the priors imposed by our SPS model
(Appendix~\ref{sec:model_priors}, Figure~\ref{fig:sfh_prior}).

Another key advantage of {\sc PROVABGS} is that it will infer galaxy
properties from joint SED modeling of photometry \emph{and spectra}. 
Our results illustrate the advantages of including spectra in SED modeling. 
Galaxy spectra provide substantial statistical power for constraining 
galaxy properties. 
In addition to tightening constraints overall, their statistical power is
essential for mitigating the effect of the model priors. 
For instance, including spectra in the SED modeling significantly reduces the
bias of our $\zmw$ and $\tage$ constraints (Figure~\ref{fig:etas}). 
It also reduces the lower bound on the inferred $\avgsfr$. 
In fact, without spectra, we are dominated by priors on $\avgsfr$ and cannot
robustly infer galaxy properties of quiescent galaxies with $\log\avgsfr < 0$
dex.
%We emphasize that all of these benefits come from spectra with the SNR of BGS, which is observing during bright time. 

{\sc PROVABGS} will be a value-added galaxy catalog with unprecedented
statistical power. 
With physical galaxy properties of over $10$ million DESI BGS galaxies, 
{\sc PROVABGS} will provide a transformational galaxy sample to extend
previous statistical galaxy studies. 
For example, we will be able to make the most precise measurement of the
stellar mass function~\citep[SMF]{li2009, moustakas2013}, star-forming
sequence~\citep{noeske2007}, mass-metallicity relation~\citep{tremonti2004}, or
any other summary statistic of galaxy populations. 
{\sc PROVABGS} will also include large sample of dwarf galaxies thanks to the
faint apparent magnitude limit of BGS. 
Dwarf galaxies are dark matter dominated and, thus, probe the physics of
dark matter; they are also sensitive to star formation feedback and can help
distinguish different aspects of galaxy formation. 
Galaxy studies examining the galaxy-halo connection can also be extended to
exploit the additional statistical power of {\sc
PROVABGS}~\citep[\emph{e.g.}][]{tinker2011, wetzel2013, zu2015, hahn2017,
hahn2019b}. 
With detailed galaxy properties, {\sc PROVABGS} will also enable
multiple-tracer galaxy clustering analyses that can circumvent cosmic variance
in inferring cosmological parameters~\citep{seljak2009, mcdonald2009,
wang2020}.
Analyses exploiting new forward modeling approaches, such as \cite{hahn2021},
will also greatly benefit from the statistical power of {\sc PROVABGS}.
\ch{Text above will be updated with more science applications based on any
feedback.}

In addition to the applications above, {\sc PROVABGS} will also unlock
applications that can exploit the full posteriors that the catalog will
provide. 
In this work, we utilized the posteriors in order to quantify accuracy and
precision of galaxy population constraints using population inference with a
hierarchical Bayesian approach. 
This is only the \emph{simplest} illustration of such an approach. 
In fact, we can extend the approach to the ultimate goal of any observational
galaxy study --- to infer the full distribution of all galaxy properties given
a set of galaxy observations, $p(\theta | \{X_i\})$.
This is only possible with population inference using the full posteriors for
each galaxy.
%We can also use population inference to robustly derive galaxy property distributions of galaxy subpopulations --- \emph{without stacking observations}.
Furthermore, population inference allows us to avoid stacking observations. 
Stacking makes the strong assumption that galaxies that are grouped together in
some \emph{e.g.} color-space are from a subpopulation with the same properties. 
This assumption fails if, for instance, there are contaminants or multiple
disparate galaxy subpopulations that are degenerate in color-space and
therefore are included in the stack. 
With the {\sc PROVABGS} posteriors we can infer fully probabilistic galaxy
population statistics, which will allow us to robustly probe even the lowest
signal-to-noise regime.
A probabilistic SMF of BGS, for example, will provide accurate constraints on
the low mass end down to ${\sim}10^{7} M_\odot$ (Figure~\ref{fig:bgs_mstar}),
which has important implications for both galaxy evolution and cosmology. 
With all of the applications listed above, {\sc PROVABGS} will enable us to
fully extract the statistical power of >10 million BGS galaxies.
